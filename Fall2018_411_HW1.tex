%This is a Latex file.
\documentclass[12pt]{article}
\usepackage{latexsym,fancyhdr,amsmath,amsfonts,dsfont,amsthm,amssymb,mathrsfs,mathtools}
\usepackage[margin=1in]{geometry}
\usepackage{lastpage} % Required to determine the last page for the footer
\usepackage{tikz}
\usepackage{url}

\parindent 0pt

\pagestyle{fancy} \lhead{\sf MTH 411} \chead{\sf Homework \#1}
\rhead{\sf Due: Saturday 9/17/2018} \lfoot{} \cfoot{} \rfoot{}

\newcommand{\N}{\mathds{N}}
\newcommand{\Z}{\mathds{Z}}
\renewcommand{\vec}[1]{\overrightarrow{#1}}
\newcommand{\C}{\mathbb{C}}
\newcommand{\R}{\mathbb{R}}
\DeclarePairedDelimiter\abs{\lvert}{\rvert}

\begin{document}

{\bf Current Reading}: Sections 0 and 1. Start Section 2 if possible.

\

{\bf Section 0 Problems}: You should be able to work out all problems. However, you may skip \#18-22 (that all deal with the notion of {\it cardinality}). Problems \#18-19 are quite surprising, so try them if you'd like a challenge.

\

{\bf Section 1 Problems}: You should be able to work out all problems. Though you don't have to submit them, \#41 followed by \#38-40 form a beautiful suite. 

\hrulefill

\

The following problems are due on 11:59pm Monday 9/17.  Submit both LaTeX and pdf files to the appropriate D2L Dropbox. 

Please name the files using the following format:
\begin{center}
\fbox{Klum$\_$Austin$\_$MTH411$\_$Fall2018$\_$HW$\_$01}
\end{center}

You may discuss the problems with your classmates, but your write-up must be your own.  Any problems marked with an asterisk (*) denote problems you can not discuss with anyone except for me.\\

Please include the statements of the problems in your HW submissions. For the Extra problems you can copy the statements from the LaTeX file that generated this pdf. However, you will have to transcribe the remaining problems from Fraleigh.

\

\textbf{Section 0}: 12, 14, 26, 31, 32, 36\\
\textbf{Section 1}: 9, 19, 28, 31, 34, 35-37\\
\textbf{Extras}:
\begin{enumerate}

\item Define the binary relation $\sim$ on $\mathbb{R}$ in the following way: for $a,b \in \mathbb{R}$, we say that $a \sim b$ if\footnote{In a definition, an ``if'' is always an ``if and only if''.} $a-b \in \mathbb{Z}$. Prove that $\sim$ is an equivalence relation.

\item Let $D=\{d \in \mathbb{R} \mid \text{there is an integer } k \text{ such that } d=2^k\}$. Let $M(D)$ be the set of $3 \times 3$ matrices with real entries whose determinant is in $D$.
\begin{enumerate}
\item Prove: if $A \in M(D)$, then $A^{-1} \in M(D)$
\item Prove: if $A, B \in M(D)$, then $AB \in M(D)$
\end{enumerate}

\item Let $f : A \to B$ and $g : B \to C$. Prove: if $f$ and $g$ are bijective, then $g \circ f$ is bijective.

\item Let $T$ be the set of all $3 \times 3$ matrices with real entries. Define the function $\phi : T \to \mathbb{R}$ by the rule $\phi(A) = \sqrt{2} \det(A)$. Prove $\phi$ is surjective, but not bijective.

\item Let $M$ be the set of $2 \times 2$ matrices. For matrices $A$ and $B$, define the relation $\sim$ by saying that $A \sim B$ if $A$ and $B$ are similar\footnote{What did that mean again?} matrices. Prove that $\sim$ is an equivalence relation.

\end{enumerate}

\clearpage

\begin{enumerate}
	\item[0.12] Let $A = \{1,2,3\}$ and $B = \{2,4,6\}$ For each relation between $A$ and $B$ given as a subset of $ A \times B$, decide whether it is a function mapping $A$ into $B$. If it is function, decide whether it is one to one and whether it is onto $B$.
	\begin{enumerate}
		\item \{(1,4), (2,4), (3,6)\} \quad Yes, neither one to one or onto.
		\item \{(1,4), (2,6), (3,4)\} \quad Yes, neither one to one or onto.
		\item \{(1,6), (1,2), (1,4)\} \quad Not a function.
		\item \{(2,2), (1,6), (3,4)\} \quad Yes, one to one and onto.
		\item \{(1,6), (2,6), (3,6)\} \quad Yes, neither one to one and onto.
		\item \{(1,2), (2,6), (2,4)\} \quad Not a function.
	\end{enumerate}

		\item[0.14] Recall that for $a,b \in \R$ and $a \geq b$, the $\textbf{closed interval } [a,b]$ in $\R$ is defined by $[a,b] = \{x\in \R|a\leq x \leq b \}$. Show that the given intervals have the same cardinality by giving a formula for a one-to-one function $f$ mapping the first interval onto the second.
	\begin{enumerate}
		\item $[0,1]$ and $[0,2]$ \quad	Let the function be $f(x)=2x$
		\item $[1,3]$ and $[5,25]$ \quad Let the function be $f(x)=10(x-1)+5$
		\item $[a,b]$ and $[c,d]$ \quad	Let the function be $f(x)=(\frac{d-c}{b-a})(x-a)+c$
	\end{enumerate}
	
	\item[0.26] Find the number of different partitions of a set having the given number of elements.\quad 4 elements \\
		15 different partitions.
	\item[0.31] $x$ $\mathscr{R} $ $y$ in $\R$ if $\abs{x} = \abs{y}$
		`
	\item[0.32]  $x$ $\mathscr{R} $ $y$ in $\R$ if $\abs{x-y} \leq 3$
	
	\item[0.36] Let $n \in \Z^+$ and let $\sim$ be defined on $\Z$ by $r \sim s$ if and only if $r-s$ is divisible by $n$, that is, if and only if $r-s=nq$ for some $q \in \Z$.
	\begin{enumerate}
		\item Show that $\sim$ is an equivalence relation on $\Z$.
		\item Show that, when restricted to the subset $\Z^+$ of $\Z$, this $\sim$ is the equivalence relation, \textit{congruence modulo n} of Example 0.20
		\item The cells of this partition of $\Z$ are \textit{residue classes modulo n} in $\Z$. Repeat Exercise 35 for the residue classes modulo $n$ in $\Z$ rather than in $\Z^+$ using the notation $\{\cdots,\#,\#,\#,\cdots\}$ for these infinite sets.
	\end{enumerate}
	\item[1.09] Compute the given arithmetic expression and give the answer in the form $a + bi$ for $a,b \in \R .\quad (1-i)^5 $
		
	\item[1.19] Find all solutions in $\C$ of the given equation:\quad $z^3 = -27i$
	
	\item[1.28] Explain why the expression $5+_6 8$ in $\R_6$ makes no sense.
	
	\item[1.31] Find \textit{all} solutions $x$ of the given equation: \quad $x+_7x=3$ in $\Z_7$
	
	\item[1.34] Find \textit{all} solutions $x$ of the given equation: \quad $x+_4x+_4x+_4x=0$ in $\Z_4$
	
	\item[1.35] Example 1.15 asserts that there is an isomorphism of $U_8$ with $\Z_8$ in which $\zeta = e^{i(\pi/4)} \leftrightarrow 5$ and $\zeta^2 \leftrightarrow 2$. Find the element of $\Z_8$ that corresponds to each of the remaining six elements $\zeta^m$ in $U_8$ for $m=0,3,4,5,6,$ and 7.
	
	\item[1.36] There is an isomorphism of $U_7$ with $\Z_7$ in which $\zeta = e^{i(2\pi/7)} \leftrightarrow 4$. Find the element in $\Z_7$ to which $\zeta^m$ must correspond for $m=0,2,3,4,5,$ and 6.
	
	\item[1.37] Why can there be no isomorphism of $U_6$ with $\Z_6$ in which $\zeta=e^{i(\pi/3)}$ corresponds to 4?
	
\end{enumerate}


\end{document}
