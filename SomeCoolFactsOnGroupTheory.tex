
	\documentclass[12pt]{article}
	\usepackage{amsmath,amssymb,amsxtra}
        \usepackage{makeidx}
	%\usepackage{makeidx, color, latexsym, graphicx, maple2e}
             %\setlength{\oddsidemargin}{-.5in}
             %\setlength{\evensidemargin}{-.5in}
             %\setlength{\textwidth}{7.25in}
	    %\topmargin  -.7in
	    \textheight 8.45in
	   % \pagestyle{empty}
%\usepackage{amssymb,latexsym,amsxtra,amscd,amsfonts,amsthm,amsmath,verbatim,epsf}
%\usepackage{amssymb,latexsym,amsxtra}	    
   %%%%%%
    \newcommand{\deff}[1]{{\bf Definition: #1} }
    \newcommand{\addtoc}[1]{\addcontentsline{toc}{subsection}{#1}}
    \newcommand{\pictt}[1]{\includegraphics{#1}}
\newcommand{\nextpage}{\vspace*{\fill}\newpage}


       \newcommand{\str}[1]{#1\backup\rule[.2em]{1em}{0.02em}} 
        \newcommand{\bigfrac}[2]{\frac{\textstyle #1}{\textstyle #2}}
        \newcommand{\Lim}{\displaystyle\lim}

   	\def\bbd#1{\leavevmode\setbox0=\hbox{#1}%
   	\kern-.025em\copy0 \kern-\wd0
   	\kern .05em\copy0 \kern-\wd0
  	\kern-.025em\raise.0433em\box0
  	}
   	\def\nexto{\kern -0.54em}
   	\def\DF{ \mbox{\rm {I\ \nexto F}} }
    	\def\DN{ \mbox{\rm {I\ \nexto N}} }
    	\def\DZ{ \mbox{\rm Z \kern -0.75em Z} }
    	\def\DC{ \mbox{\rm\hbox{C \kern-0.8em\raise0.2ex
    	\hbox{\vrule height5.4pt width0.7pt} \kern0.45em}} }
    	\def\DQ{ \mbox{\rm\hbox{Q \kern-0.85em\raise0.25ex
    	\hbox{\vrule height5.4pt width0.7pt} \kern0.45em}} }
    	\def\DP{ \mbox{\rm\hbox{P\kern-0.75em
	\hbox{\vrule height9pt width0.7pt} \kern0.45em}} }
	 \newcommand{\mat}[2]{\left( \begin{array}{*{#1}{c}}#2\end{array}\right)}
 \newcommand{\matr}[2]{\left( \begin{array}{*{#1}{r}}#2\end{array}\right)}
 \newcommand{\dat}[2]{\left| \begin{array}{*{#1}{c}}#2\end{array}\right|}
 \newcommand{\datr}[2]{\left| \begin{array}{*{#1}{r}}#2\end{array}\right|}
\newcommand{\ord}{\mbox{ord}}
 \newcommand{\seg}[1]{\overline{#1}}
 \newcommand{\ray}[1]{\overrightarrow{#1}}
 \newcommand{\lin}[1]{\overleftrightarrow{#1}}
 \newcommand{\sdash}{\!\!-\!\!}
 \newcommand{\betw}[3]{#1\sdash#2\sdash#3}
\makeindex
\begin{document}
\begin{titlepage}
    \vspace*{.5in}
    \begin{center}
{\Huge\bf Notes on Group Theory}

\vspace{1in}
{\sl by Avinash Sathaye, Professor of Mathematics }


     {\today}
     \end{center}

     \nextpage
         
     \end{titlepage}

\tableofcontents
\nextpage

\section{Preparation.} Be sure to read thru the Preliminaries: pages
1-15 from the book. I will assume that you have understood them and can
do the exercises. Of course, you are free to ask questions about the
material, privately, or in class, and get it clarified.

\section{Group axioms and definitions.}

\deff{Group}\index{group!definition of}
We define a group to be a non empty set $G$ together with a {\bf binary
operation}
$f:G\times G \Rightarrow G$ such that:
\begin{enumerate}
\item $f$ is associative.\index{associative}
\item There exists an element $e\in G$ such that
$$f(e,g)=f(g,e) = g ~~~\forall g\in G.$$
\item For each $g\in G$ there is some element $g'\in G$ such that
$$f(g,g')=f(g',g)=e.$$
\end{enumerate}


\addtoc{Shortcuts.}
{\bf We shall take the following shortcuts.}
\begin{enumerate}
\item \deff{commutative group} A group $G$ is said to be commutative if
$f(g,h)=f(h,g)$ $\forall g,h \in G$. Such a group is also called {\bf
abelian}.\index{group!commutative or abelian}

\item We shall rewrite $f(g,h)$ as $g+h$ if the group is abelian and
$g\cdot h$ or just $gh$ in general, dropping all reference to ``$f$''.
On rare occasions, for comparing different operations, we may use other
symbols to denote the operation.

\item {\bf Extended product notation.}\index{group!product notation extended}
Suppose we are using the
convention to write the product of elements $g,h$ as simply $gh$.
We shall extend this to subsets $S,T\subset G$ as follows.

By $ST$ we shall denote the set $\{st | s\in S,t\in T\}$.

We shall extend this to products of several sets as needed.

If a set is a singleton, then we may simply write the element in place
of the set. Thus ${s}T$ can be shortened to $sT$.

We may also use the inverses of sets, thus $S^{-1}=\{s^{-1}|s\in S\}$.
We could also use powers $S^2 = SS$ etc., but we prefer not to take this
shortcut!

\item $f(g,g)$ will become $g^2$ with no mention of ``$f$''. This is
naturally extended to $g^n$. The same expression becomes $ng$ in case
the group is abelian and we are using ``$+$'' as the operation symbol.

\item The element $e$ is proved to be unique and is called the
{\bf identity element}\index{group!identity} of the group.
It may be  denoted by $0$ for an abelian group and by $1$ or $e$  in general.

\item The associated element $g'$ is also shown to be unique and is
called the inverse of $g$. It is denoted by the more suggestive notation
$g^{-1}$ which is replaced by $-g$ when the group is abelian and the
operation symbol is $+$.
\index{group!inverse of an element}



\end{enumerate}


\subsection{Cyclic groups.}


\deff{Subgroup}\index{group!subgroup}
Suppose that $G$ is a group with binary operation $f$.
\begin{itemize}
\item
Suppose that $H\subset G$ is a non empty subset which is closed under
the binary operation $f$, i.e. $f(h_1,h_2)\in H$ whenever $h_1,h_2\in
H$.
\item
Also suppose that for every $h\in H$ the inverse $h^{-1}$ also
belongs to $H$.
\end{itemize}

Then it is easy to see that the restriction of the operation $f$ to $H$
makes it into a group and $H$ is said to be a subgroup of $G$.

We shall write this in notation as $H<G$. 
{\bf Caution!} Note that this notation may lead you to believe that $H$
has to be smaller than $G$, so $H$ cannot equal $G$. That is not the
implication! Many authors on group theory avoid this notation, perhaps
to avoid this confusion!

\deff{Order of an element}\index{group!order of an element}
If $G$ is a group and $g\in G$ then we
define:
$$\mbox{ order of } g ~=~ |g| = \min\{ n | g^n = e\}.$$
Remember the convention that the minimum of an empty set is $\infty$.

We shall use the notation:
$$<g> = \{ g^n | n\in \DZ \}$$
and call it the {\bf cyclic group} generated by $g$. Note that we are
tacitly claiming that it is a subgroup of $G$, i.e. $<g><G$. 

{\bf Alternatively} we could define $<g>$ as the smallest subgroup of $G$
which contains $g$.

Also note that if $n=-m$ a negative integer, then by $g^n$ we mean
$\left(g^{-1}\right)^m$. It is easy to note that $\left(g^{-1}\right)^mg^m=e$ and
hence $\left(g^{-1}\right)^m = \left(g^m\right)^{-1}$.

\index{group!cyclic}

It can be easily show that the set
$<g>$ is an infinite set if $|g|=\infty$ and has
exactly $d$ elements if $|g|=d$.

Moreover, if $|g|=d$ is finite, then this set $<g>$ can be explicitly
written as $\{ e=g^0,g=g^1, g^2,\cdots,g^{d-1}\}$.

Further, for any $n\in \DZ$, we can prove that $g^n = g^r$ where $r$ is
the remainder when $n$ is divided by $d$.

{\bf Observation.}
If the elements of $H$ are of finite order, then the condition about the
inverses can be omitted, since the inverse of an element with order $d$
is seen to be its $(d-1)$-th power.

Note that the cyclic group generated by a single element $g\in G$ can be
also characterized as the smallest subgroup of $G$ containing $g$.
Equivalently, it can be also defined as 
$$<g> = \bigcap\{H ~|~ g \in H < G\}$$
or the intersection of all subgroups of $G$ containing $g$. 

This observation can be strengthened to the 

\deff{Group generated by a set}.
Given a set $S\subset G$, we define a subgroup of $G$ generated by $S$
as:
$$<S>= \bigcap \{ H ~|~ S \subset H < G\}.$$

For calculation purposes, it can also be characterized as a set of
finite products
$s_1\cdots s_r$ with $s_i\in <t_i> $ where $t_i\in S$ for $i=1,\cdots,
r$.

Despite the simple definition, the actual calculation of the resulting
group can be quite complicated!

To prove these observations, you only need to know that an arbitrary
intersection of subgroups of $G$ is also a subgroup of $G$. This is
easy to prove!

\subsubsection{Exercises on  groups.}

\begin{enumerate}
\item For any subset $S\subset G$ {\bf recall:}
$$SS= \{s_1s_2 ~|~ s_1,s_2\in S\}.$$

Prove that $S<G$ iff $SS=S$ and for every $s\in S$, we have $s^{-1}\in
S$.

As before, if every element of $S$ has a finite order, then $S<G$ iff
$SS=S$.

In particular, if $G$ is a finite group, then $S<G$ iff $SS=S$.

\item Given elements $g,h\in G$ the element $ghg^{-1}$ will turn out to
be very useful. We shall call it the conjugate of $h$ by $g$. It may be
denoted by the suggestive notation $h^g$.

Prove that $|h| = |ghg^{-1}|$.

Deduce that $|xy|=|yx|$ for any $x,y\in G$.

Note that the equations are supposed to work naturally even for infinite orders.

\item {\bf Conjugate subgroups.}


Prove that if $H<G$ and $g\in G$, then $gHg^{-1} <G$.

For proof, simply note that $(gHg^{-1})(gHg^{-1}) = gHHg^{-1} =
gHg^{-1}$, the last equation follows since $H<G$.

Also, it is obvious that $gHg^{-1}$ contains the inverses of all its
elements, since 
$$\left(ghg^{-1}\right)^{-1}= gh^{-1}g^{-1}\in gHg^{-1}$$
since $H<G$ implies $h^{-1}\in H$ when $h\in H$.

The group $gHg^{-1}$ is said to be a conjugate of $H$ by $g$ and is also
written in the suggestive notation $H^g$.
\index{group!conjugate of a subgroup}

\item 
This calculation also leads to {\bf the most important concept} in group
theory.

\deff{Normal subgroup.}\index{group!normal subgroup}
A subgroup $H<G$ is said to be a normal subgroup if $gHg^{-1}=H$ for all
$g\in G$.

We shall use the customary notation $$H\vartriangleleft G$$ to indicate 
that $H$ is a normal subgroup of $G$.\index{group!normal
subgroup!notation for}

Prove that the subgroups $<e>$ and $G$ are always normal.
These are also called the {\bf trivial subgroups} for obvious reasons.
\index{group!trivial subgroup}


\deff{Simple group.}\index{group!simple}
A group $G$ is said to be simple if $<e>$ and $G$ are the only normal
subgroups, i.e. $G$ has only trivial normal subgroups.

One of the biggest achievements of group theory is considered to be the
{\bf classification of all finite simple groups}.
\footnote{Despite the announcement of this great theorem some years
back, some doubts still persist about the complete validity of this
theorem. The number of mathematicians in the world who may know the
complete proof of the theorem is rather small and people are still
working on writing the definitive version. 

Many group theorists are said to have left group theory after the
announcement of the theorem, considering that there was not much left to
prove!}

\item Prove that if $|g|=d<\infty$ and $n$ is any integer,
then:
$$|g^n| = \frac{d}{\mbox{GCD}(n,d)}.$$

Make and prove an appropriate statement if $|g|=\infty$.

\item  Deduce that if $|g|=d<\infty$ and if $h=g^n \in <g>$ then 
$$<h>=<g> \mbox{ iff } \mbox{GCD}(n,d)=1.$$

Make and prove an appropriate statement if $|g|=\infty$.
\item 
We shall say that {\bf an element $\mathbf{g\in G}$ generates $\mathbf{G}$} if
$G=<g>$, i.e. $G$ is a cyclic group consisting of all powers of $g$.
\index{group!generator of}

Let $n$ be a positive integer and define $\DZ_n$ by:
$$\DZ_n = \DZ/n\DZ.$$
Prove that $\DZ_n$ is an abelian group under the usual ``$+$'' inherited from $\DZ$.
Prove that it is a cyclic group.

Prove that $\overline{a}\in \DZ_n$ is a generator of $\DZ_n$ iff $\mbox{GCD}(a,n)=1$.

Deduce that the number of distinct generators of $\DZ_n$ are exactly
$\phi(n)$ the Euler $\phi$ function evaluated at $n$.

\item {\bf Commutativity in groups.}
\begin{itemize}
\item Prove that for any elements $g,h\in G$ we have
$(gh)^{-1}=h^{-1}g^{-1}$. More generally $(g_1g_2\cdots g_r)^{-1} =
g_r^{-1}\cdots g_2^{-1}g_1^{-1}$.

\item We say that elements $g,h\in G$ commute with each other if
$gh=hg$.\index{group!commuting elements}

Prove that $g,h$ commute with each other iff $ghg^{-1}h^{-1}=e$.
Thus the element $ghg^{-1}h^{-1}$ measures how far the elements don't commute!

We define the {\bf Commutator } of elements $g,h\in G$ to be the
element $ghg^{-1}h^{-1}$. We shall denote it by the symbol $[g,h]$.
\index{group!Commutator of elements}

The group generated by all the commutators of elements of $G$ 
is called the {\bf Commutator subgroup} of $G$.\index{group!Commutator subgroup}
It is often denoted by the convenient symbol $G'$, or by a more
suggestive $[G,G]$.

{\bf Caution:} Don't forget that elements of $[G,G]$ are products of
Commutator elements and cannot, in general, be written as a single
Commutator.

Thus $gh=hg$ iff $[g,h]=e$.

{\bf Note:} We could write $[g,h]=h^gh^{-1} = g(g^{-1})^h$. This can be useful
later!

\item Prove the formula $[g,h]^{-1} = [h,g]$.
\item Prove that if a group has the property that $|g|\leq 2 $ for each
$g\in G$ then $G$ is abelian.

{\bf Hint:} Try to show that $[g,h]=e$ for
all $g,h\in G$.

\item {\bf Challenging problem!}

Let us say that a group has property $P_m$ if $(gh)^m = g^mh^m$
for all $g,h\in G$.

Note that $G$ is abelian iff $G$ has property $P_m$ for each
$m=1,2,3,\cdots$.

Prove that $G$ is abelian iff there is some non negative integer $n$
such that $G$ has property $P_n,P_{n+1},P_{n+2}$.

{\bf Hint:} Note that $P_0$ and $P_1$ are trivially true.
Prove that $P_2$ alone implies that $G$ is abelian.

Thus, the first non trivial case is $n=3$.

\item {\bf Another challenging problem!}

Suppose we replace our group axioms by the following apparently weaker
ones:
\begin{itemize}
\item $G$ is a non empty set with a binary {\it associative} 
operation denoted by
``$\cdot$''.
\item $G$ has an element $e$ such that $a\cdot e =a ~~\forall a\in G$.
\item Each element $a\in G$ has an associated element denoted as $h(a)$
such that $a\cdot h(a) = e$.
\end{itemize}

Prove that $(G,\cdot )$ is a  group! So, the axioms are not weaker at
all, despite their appearance!

Now here is a shocker. Suppose that we keep the first two axioms as they 
are but replace the third axiom by a similar but different one thus:

Each element $a\in G$ has an associated element denoted as $h(a)$
such that $h(a)\cdot a = e$.

Find an example $(G,\cdot)$ where the new axioms are satisfied, but $G$ is not a
group!
\end{itemize}
\end{enumerate}

\subsection{Permutation groups.}\index{group!permutation group}
A more useful way of thinking about a group is this.

Fix some set $\Omega$. {\bf Define} $S_\Omega$ to be the set of all 
bijective functions from $\Omega$ to itself. If the set $\Omega$ is
finite, then these are easily viewed as permutations of the set. The
same can be thought for infinite sets with a suitable imagination.

The set $S_\Omega$ with the binary operation of composition of the
functions forms a natural group. The identity is the identity function
and is often denoted as $I$ or $Id$. The inverse is simply the inverse
function.

Later on, we shall prove a theorem which says that all groups can be
viewed as subgroups  
of $S_\Omega$ for some $\Omega$.

For a finite set $\Omega$, it is convenient to develop efficient
notation and conventions. Since most of our groups will be finite, this 
will be very useful.

\subsubsection{Conventions for a finite permutation group.}
Here are the usual simplifications.
\begin{enumerate}
\item
If $\Omega$ has $n$ elements, we agree to list them as simply the
natural numbers $1,2,\cdots,n$ and simplify the notation $S_\Omega$ to
 $S_n$.

The permutation functions are then best described by simply listing 
the images of various elements of $\Omega$ in the form of a table. Example:
$$\sigma =
\matr{4}{1 & 2 & 3 & 4 \\
         3 & 2 & 1 & 4 \\}
         $$

\item While this is efficient, it has two drawbacks. We keep on writing
the same first row of $n$ numbers for every element and we probably
should 
economize on the second row.

For example, this $\sigma$ above can be quickly described as 
``exchange $1$ and $3$, leave the rest unchanged''.

In symbols, we shall agree to write it as $\matr{2}{1 & 3}$ which tells
us about swapping $1,3$ and we agree that the unmentioned ones are
unchanged!

More generally, we shall agree that $\matr{3}{2 & 3 & 4}$ shall be the
permutation which send $2$ to $3$, $3$ to $4$ and $4$ to $2$. It then
leaves $1$ unchanged. 

We formalize this thus:

{\bf Cycle notation} A sequence 
$\matr{5}{a_1 & a_2 & \cdots & a_{r-1} & a_r}$
 is called
a cycle of length $r$ (or simply an $r$-cycle)  and intended to permute the elements 
$\matr{5}{a_1 & a_2 & \cdots & a_{r-1} & a_r}$ in a cycle.
\index{group!cycle notation}


\item {\bf Combining cycles}
In permutation groups we often omit the composition symbol and simply
write cycles next to each other, with the understanding that each cycle
represents a function and these are to be composed from right to left.

Thus $\matr{2}{1 & 3}\matr{2}{2 & 4}$ is the same as the permutation
$$\matr{4}{1 & 2 & 3 & 4\\
           3 & 4 & 1 & 2\\}.$$

When the cycles have no common elements, then they commute, but not
otherwise.

It is a theorem that every permutation can be written as a product of
disjoint cycles and the expression is unique, except for the order of
the cycles and the fact that entries of a cycle can be permuted in a
cyclic manner without changing its meaning.

An example of permuting the entries in a cycle is:
$$\matr{3}{1 & 2 & 3 } = \matr{3}{ 2 & 3 & 1 } = \matr{3}{3 & 1 & 2 }$$

\item {\bf Orders of permutations.}\index{permutations!order calculations}
It is easy to check that the order of an $r$-cycle is simply $r$.
If a permutation $\sigma$ is equal to a product $\sigma_1 \sigma_2
\cdots \sigma_m$ of $m$ {\bf mutually disjoint cycles}, then it is easy to see
that
$$\sigma^i = \sigma_1^i \sigma_2^i \cdots \sigma_m^i$$
and indeed, this expression gives a disjoint cycle representation
(after breaking up the powers separately into cycles if needed).

Then we can easily see that
$$\sigma^n = Id \mbox{ iff } \sigma_i^n = Id ~~\forall i=1,\cdots,m.$$
Thus clearly $\sigma^n=Id$ iff
$\mbox{LCM}(|\sigma_1|,|\sigma_2|,\cdots,|\sigma_m|)$ divides $n$.

In other words
$|\sigma|=\mbox{LCM}(|\sigma_1|,|\sigma_2|,\cdots,|\sigma_m|)$.

{Warning:} If the cycles are not disjoint, then the formula is not even
expected to work! 
For example,
$$|\matr{2}{1 & 2} \matr{2}{1 & 3}| = |\matr{3}{1 & 3 &  2}| = 3$$
but the LCM of the orders of the separate cycles is only $2$.
\end{enumerate}

\subsubsection{Exercises on Permutations.}
\begin{enumerate}
\item {\bf Conjugation.} Given two permutations $\sigma$ and $\tau$ we
observe the following fact for the
conjugate\index{permutation!conjugation trick}
$\theta = \tau \sigma  \tau^{-1}$:

$$\theta(\tau(i))= \tau \circ \sigma\circ \tau(\tau^{-1}(i))= \tau(\sigma(i)).$$

Thus, the easiest way to describe $\theta$ is to take the display of
$\sigma$ and if it has $i\rightarrow \sigma(i)$ act by $\tau$ on both
the entries to write $\tau(i) \rightarrow \tau(\sigma(i))$.

The best way to understand this is to work this example:

Let $\sigma=\matr{3}{1 & 2 & 3} \matr{2}{4 & 5}$ and $\tau =
\matr{2}{1 & 4}\matr{2}{2 & 5}$.

Then prove that 
$$\sigma^{\tau} = \tau \sigma \tau^{-1} = 
\matr{3}{\tau(1) & \tau(2) & \tau(3)} \matr{2}{\tau(4) & \tau(5)} 
= \matr{3}{4 & 5 & 3} \matr{2}{1 & 2}.$$

Verify this with the observation as well as with direct computation.
{\bf This justifies the notation $\mathbf{\sigma^{\tau}}$, since it is 
$\mathbf{\tau}$ acting on $\mathbf{\sigma}$.}

Similarly, show that 
$$\tau^\sigma = \matr{2}{2 & 5} \matr{2}{3 & 4}.$$

\item Use the above to give another proof that a permutation has the
same order as that of any of its conjugate.

It is useful define the {\bf type of a
permutation}\index{permutation!type of}
to be a sequence of pairs $[r,s]$ if the permutation contains $s$ disjoint
$r$-cycles in its cycle representation. To make the definition
meaningful, we list the cycle lengths in decreasing order. Also, for
convenience, we shall drop the ``$1$-cycles'' from consideration!
\footnote{{\bf Be aware} that some people might choose to keep them in notation, so that
the sum of the terms $rs$ is always equal to $n$.}

Thus our $\sigma$ above has type $([3,1],[2,1])$ while the $\tau$ above  
has type $([2,2])$.

As a convention, we define the type of identity to be $([0,0])$.
This is a technical convenience!

Prove that a permutation has the same type as any of its conjugates.
Also, prove that the order of a permutation of type
$([r_1,s_1],\cdots,[r_m,s_m])$ is $\mbox{LCM}(r_1,\cdots,r_m)$.

\item Define the function 
$$M(n) = \max\{|\sigma| ~|~ \sigma\in S_n\}.$$

Determine the values of $M(n)$ for $n=1,2,\cdots,10$.

{\bf Challenge:} Can you make a formula for $M(n)$ in general?

\item Suppose that $\sigma$ is of type $([r,1])$, i.e. is an $r$-cycle.

Prove that for every $i=1,2,\cdots$ we have that $\sigma^i$ has type
$([a,b])$ where $ab=r$ or $\sigma^i=Id$. 
In words, this says that a power of a cycle
breaks up into a certain number of cycles of equal length.

{\bf Hint:} Identify the formula for cycles of small length and then
guess. Try cycles of lengths $5,6,9,10,12$.

\item Prove that a permutation is of order $2$ iff it has type $([2,s])$
for some $s$.

\item {\bf Conjugacy classes.}\index{group!conjugacy class}
The conjugacy class of an element $g\in G$ is said to be 
$\{g^h ~|~ h \in G\}$.
The number of elements in a conjugacy class is going to be an important
concept later on.

Prove that the conjugacy class of a permutation
$\sigma\in S_n$ is the set of permutations with the same type as
$\sigma$.

For $\sigma = \matr{3}{1 & 2 & 3} \matr{2}{4 & 5} \in S_5$, 
determine its conjugacy class explicitly. 

{\bf Note that the class depends on the group, so for the same 
$\mathbf{\sigma}$ 
the class is bigger if we work in $\mathbf{S_6}$.}

\item Let $n\geq 4$.

Prove that the conjugacy class of 
$\tau = \matr{2}{1 & 2} \matr{2}{3 & 4}\in S_n$
has $\frac{n(n-1)(n-2)(n-3)}{8}$ elements.

{\bf Hint:} You are counting all permutations of type $([2,2])$. 


\end{enumerate}

\subsection{Matrix Groups.}
The Euclidean or affine spaces $\Re^n$ as well as the general vector
spaces $F^n$ over a field $F$ are extremely important in many branches
of mathematics as well as applications.

These spaces naturally give rise to a nice collection of groups which
described the ``change of coordinate'' transformations. These can also
be thought of as invertible transformations on the underlying set of
points with certain extra restrictions, depending on the kind of
geometry being studied.

Rather than get into the details of these spaces, we shall concentrate on
the underlying groups as follows. 

\begin{enumerate}
\item{\bf Field}\index{field} 
A field is an abelian group under an addition operation with
identity $0$ together with a commutative multiplicative group structure on 
its non zero elements, conveniently denoted by $F^\times = \{ a\in F | a \neq
0\}$.

Further the addition and the multiplication operations satisfy the
distributive law:
$$a(b+c)=ab+ac ~~\forall a,b,c \in F.$$

\item{\bf General Linear Group}\index{group!general linear} 
The set of all $n\times n$-matrices with
entries in $F$ and a non zero determinant is denoted by $GL_n(F)$.

Abstractly, the group can be thought off as the set of invertible linear
transformations of an $n$-dimensional vector space over $F$.
The columns of the matrix, then can be thought of as the images of standard
basis vectors, if we identify the vector space with columns of vectors
with $n$-entries.

The identity of $GL_n(F)$ is the identity matrix $I_n$.

Many people use a simpler notation $GL(n,F)$ with a similar change for
the other notations below.

We shall have use for the {\bf Special Linear Group}
\index{group!special linear} $SL_n(F)$ which
consists of those matrices in $GL_n(F)$ which have determinant $1$.
It is easy to see that $SL_n(F)<GL_n(F)$.

An $n\times n$ matrix $A$ is said to be {\bf elementary}
\index{matrix!elementary} 
if its entries are
the same as that of $I_n$, except for one entry, say $A_{ij} = c$ with
$i\neq j$. (This means $A_{ij}$ is allowed
to be non zero.) We shall denote such a matrix by $E_{ij}(c)$.

We shall define $E_n(F)$ to be the group generated by the various
$E_{ij}(c)$. As before, this can be simply thought of as the smallest
subgroup of $GL_n(F)$ which contains all the elementary $n\times n$ matrices.
\footnote{As before, we {\bf caution} the reader that a product of elementary
matrices cannot usually be written as a single elementary matrix. It is
customary to define a product of elementary matrices to be also an
elementary matrix, but don't let it confuse you!}


It is evident that all elementary matrices are in $SL_n(F)$ and it is an
important theorem that they generate $SL_n(F)$ in the sense that every
element of $SL_n(F)$ is a product of elementary matrices!
\footnote{At higher level of group theory, we may replace the field $F$ by a
commutative ring and investigate corresponding theorems, which lead to
very important and interesting concepts!}

\item{\bf Finite fields}\index{field!finite}
Of vital interest in group theory and many applications, including
Engineering, is the concept of finite fields. The simplest examples of
finite fields are $\DZ/n\DZ$ or what we called $\DZ_n$ earlier.

The field axioms force the value of $n$ to be a prime number and it is
called the characteristic of the field.

Explicitly, we make the \deff{Characteristic of a
field}\index{field!characteristic}
$$char(F) = \min\{m | m\cdot 1 =0 \in F\}.$$

Here, by $m\cdot 1$ we mean the sum $1+\cdots+1$ with $m$ terms.
If no such $m$ exists (as in the case of the usual real field $\Re$),
then we define $char(F)=0$.
\footnote{You may object, saying that this does not conform with our
earlier convention of the minimum of an empty set being infinity!
 However, the problem is only with our wording! If we
define it as the GCD of the set of all $n$ such that $n\cdot 1=0$ then
in the real field, the GCD comes out to be $0$.}

It is proved that a finite field $F$ contains a unique $\DZ_p$ where $p$ is
its characteristic. This $\DZ_p$ is said to be its prime field.
For characteristic zero, the prime field, the field is always infinite
and contains the prime field $\DQ$.\index{field!prime field}

Moreover, a finite field $F$  is an $n$-dimensional vector space over
its prime field $\DZ_p$ and hence has $p^n$ elements.
It is customary to write $q$ for the power $p^n$ and the book denotes the
finite field as $\DF_q$. 
\footnote{Other conventional notations are $GF(n,p)$ or
$GF(q)$ where $GF$ is short for Galois Field in honor of E. Galois.}

\end{enumerate}
\subsubsection{Exercises on Matrices.}
\begin{enumerate}
\item In $GL_n(F)$ we make a:
\index{matrix!diagonal}\index{matrix!scalar}

\deff{A diagonal matrix} is an $n \times n $  matrix $M$ such that
$M_{ij}=0$ if $1\leq i,j \leq n$ and $i \neq j$. In other words, its 
non zero entries are all on the main diagonal.

Note that if $M\in GL_n(F)$, then none of the $M_{ii}$ are zero.

A diagonal matrix of the form $d\cdot I_n$ is said to be {\bf a scalar
matrix.}
where we are using the scalar multiplication by $d\in F$.

Note that a $2\times 2$ matrix $D=\matr{2}{a & 0 \\ 0 & b}$ with $ab=1$
is in $SL_2(F)$, since its determinant is $ab=1$.

Prove that $D$ is a product of elementary $2\times 2$ matrices.
{\bf Hint:} Try to multiply it by a sequence of elementary matrices to
convert it to $I_2$. Recall that this amounts to making elementary row
transformations.

If you recall the so-called LU-decomposition from your Linear Algebra,
this calculation will put you on track to proving that every member of
$SL_2(F)$ is a product of elementary matrices. 

Usually we define $E_n(F)$ to be the set of all products of elementary 
matrices in $GL_n(F)$. This is seen to be a group and indeed proved to
be equal to $SL_n(F)$.

\item {\bf The Heisenberg Group}
We can extend the book definition by setting $H_n(F)$ to be the set of
$n \times n$ unit upper triangular matrices, i.e. matrices $M$ which
satisfy:
\begin{itemize}
\item $M_{ii}=1$ for each $i=1,\cdots,n$.
\item $M_{ij}=0$ if $i>j$.
\end{itemize}

Prove that every element of $H_n(F)$ is a product of upper triangular
elementary matrices, i.e. matrices $E_{ij}(c)$ with $i<j$ and $c\in F$.
{\bf Hint:} Imitate the Gauss Elimination proof.

In particular, you get that $H_n(F)$ is a subgroup of $E_n(F)$.


\end{enumerate}

\subsection{Dihedral groups.}
One of the simplest and ubiquitous examples of non abelian groups are
the so-called dihedral groups.
These are given by:
\deff{Dihedral group}\index{group!dihedral}
A group generated by elements $r,s$ satisfying the relations 
$$s^2=e=r^n,~~ r^s=r^{-1}$$
is said to be a dihedral group of order $n$.
It is denoted by the symbol $D_{2n}$.

{\bf Important.} We shall follow the convention in the book and assume
that $n\geq 3$ for a dihedral group.
The group, as defined, even makes sense for $n=1,2$ but it comes out to
be a less interesting abelian group in each case. 

{\bf Caution:} Be aware that some books use the notation $D_n$ for the
same group.

The group can be explicitly listed as 
$$D_{2n} ~=~ \{ e,r,r^2,\cdots,r^{n-1},s,sr,sr^2,\cdots,sr^{n-1}\}.$$

In short, its elements can be listed as $s^ir^j$ where $i=0,1$ and
$j=0,1,\cdots,(n-1)$.
It is easy to explicitly describe the product of any two elements 
$$s^ir^js^lr^m = s^pr^q$$
as follows:
\begin{itemize}
\item If $l=0$ then $p=i$ and $q$ equals the remainder of $m+j$ modulo
$n$.
\item If $l=1$, then $p$ is the remainder of $i+l$ modulo $2$ and $q$ is
the remainder of $m-j$ modulo $n$.
\end{itemize}

This can be established by induction.

\subsubsection{Exercises on dihedral groups.}
{\bf As stated above, let} $D_{2n}$ be the dihedral group of order $2n$ with
$n\geq 3$.

\begin{enumerate}
\item Prove that $<r> \vartriangleleft D_{2n}$, i.e. $<r>$ is a normal
subgroup of the dihedral group $D_{2n}$.
\item Prove that $<s>$ is {\bf not a normal} subgroup of $D_{2n}$. (You will
need to use $n\geq 3$.)

\item Prove that every element of $D_{2n}$ outside $<r>$ is of order
$2$. {\bf Hint:} Compute $(sr^i)^2$ from the known formula.

\item For any $i$, prove that  $[s,sr^i] = r^{2i}$.
Deduce that $r^i$ commutes with $s$ iff $2i \equiv 0 \mod{n}$.

\item Determine all elements of $D_{2n}$ which commute with every
element of $D_{2n}$, i.e. find all $z\in D_{2n}$ such that $[z,s^ir^j]=e$
for all $i=0,1$ and $j=0,1,\cdots,(n-1)$.

First observe that it is enough to check the conditions:
$$[z,r]=[z,s]=e.$$

By calculating the commutators explicitly, show that the only
possibility for $z$ is $z=e$ or when $z=r^k$ with $n=2k$.

You may wish to prove these formulas first:
$$[sr^p,r] = r^{-2}~,~ [r^p,r]=e~,~[r^p,s]=r^2p.$$


\end{enumerate}
\subsection{Homomorphisms and Isomorphisms.}\index{group!homomorphism}
\index{group!isomorphism}
We have discussed many ways of looking at the groups. So, it becomes
essential to decide when a group is essentially the same as another.

For this, we need two main definitions.

\deff{Homomorphism of  group.}\index{homomorphism!group}
Given groups $G,H$ and a map $\phi :G\rightarrow H$ we say that $\phi$ is a
group homomorphism if it satisfies the condition:
$$\phi(g_1g_2) = \phi(g_1)\phi(g_2)~~\forall g_1,g_2\in G.$$

These observations are useful and immediate.
\begin{itemize}
\item Here, we have used no group operation symbol in $G$ as well as
$H$. It may have to be used, especially if $G,H$ happen to be the same
sets!
It is stated in the book with separate operation symbols to clarify this
point.

\item The total image (usually called the range of $\phi$) is easily 
seen to be a subgroup of $H$.

Here are the steps for this:
\begin{itemize}
\item For convenience, write $e_G$ and $e_H$ for the identities in $G$
and $H$ respectively.
>From $e_Ge_G=e_G$, deduce that $\phi(e_G)\phi(e_G)=\phi(e_G)$. This
shows that $\phi(e_G)=e_H$ (why?)
\item By calculating $\phi(gg^{-1})=\phi(e_G)$ as
$\phi(g)\phi(g^{-1})=\phi(e_G)=e_H$, deduce that
$\phi(g)^{-1}=\phi(g^{-1})\in \phi(G)$.

\item Now it is easy to deduce that $\phi(G)$ is actually a group and a
subgroup of $H$.
\end{itemize}
\item We make a
\deff{Kernel of a homomorphism}\index{homomorphism!kernel of}
The set $\{g\in G ~|~ \phi(g)=e_H\}$ is
said to be the kernel of the homomorphism $\phi:G\rightarrow H$.

We shall use the notation $Ker(\phi)$ to denote this set.
The most important property of the kernel is that it is a normal
subgroup of $G$.

The verification of the subgroup condition is easy. Now if $g\in G$ and
$k\in Ker(\phi)$, then note that
$$\phi(gkg^{-1}) =
\phi(g)\phi(k)\phi(g^{-1}) = \phi(g)e_H\phi(g)^{-1}=e_H$$
and thus $k^g\in Ker(\phi)$ whenever $k\in Ker(\phi)$.
This proves the normality.

Later on, we shall see how creating a suitable homomorphism
helps us locate a
normal subgroups of a group.

\end{itemize}

Now we make the next \deff{Isomorphism of a
group}\index{isomorphism!group}
Given groups $G,H$ and a map $\phi :G\rightarrow H$ we say that $\phi$ is a
group isomorphism if $\phi$ is an isomorphism which is both injective and surjective.

Typically, the definition of injectivity requires one to prove that
every pair of distinct elements has distinct images. However, the
homomorphism hypothesis says that a homomorphism $\phi$ is injective iff
$Ker(\phi)=\{e_G\}$.

{\bf Proof.}
Note that
$$\begin{array}{cl}
& \phi \mbox{ is injective }  \\
\mbox{ iff } & \phi(g_1)=\phi(g_2) \mbox{ implies } g_1=g_2\\
\mbox{ iff } &  \phi(g_1g_2^{-1})=e_H \mbox{ implies } g_1g_2^{-1}=e_G
\\
\mbox{ iff } & g\in Ker(\phi) \mbox{ implies } g=e_G \\
\mbox{ iff } & Ker(\phi) = \{e_G\}\\
\end{array}
$$

If $G=H$ then we say that an isomorphism from $G$ to $H$ is  {\bf an
automorphism.}\index{automorphism!group}

\subsubsection{Exercises on Homomorphisms.}
\begin{enumerate}
\item Consider two groups $\DZ_n,\DZ_m$ in our usual notation.
The notation $\overline{a}$ used so far is not sufficient since for the
same integer $a$, we need to distinguish between the two barred elements
in the different groups.

So, we shall make a better notation $[a]_n$ to denote the
$\overline{a}\in \DZ_n$ and similarly $[a]_m$ for $\overline{a}\in
\DZ_m$.

With this in mind, answer the following questions.
\begin{enumerate}
\item For any $[x]_n\in \DZ_n$, we have $|[x]_n|$ divides $n$.

\item If $F:\DZ_n\rightarrow \DZ_m$ is a homomorphism and $F([1]_n) = [a]_m$ then
$|[a]_m|$ divides $n$.

\item Deduce that if $\mbox{GCD}(m,n)=1$ then there is only one possible
homomorphism
$F:\DZ_n\rightarrow \DZ_m$, namely the zero homomorphism, i.e.
$F([x]_n)=[0]_m$ for all $x\in \DZ$.

\item Suppose that  $\mbox{GCD}(m,n)=d>1$ and write $m=bd$.

Then we can define a homomorphism
$F:\DZ_n\rightarrow \DZ_m$ by  $F([x]_n)=[bx]_m$.

{\bf Hint:} The main problem in defining a homomorphism is to verify
that it is well defined.

This means to prove that $[x]_n=[y]_n$ implies $[bx]_m=[by]_m$.

The homomorphism condition itself is trivial for all such definitions.

Determine all possible homomorphisms from $\DZ_n$ to $\DZ_m$ in this
situation (i.e. the situation of  GCD being bigger than $1$).

\item Prove that if $n\neq m$ then there is no isomorphism from $\DZ_n$
to $\DZ_m$.
\end{enumerate}

\item Suppose that $G$ is a cyclic group of finite order $n>1$ generated by $x\in
G$.
Let $t$ be any integer and consider the map $F_t:G\rightarrow G$ defined
by $F_t(y)=y^t$ for any $y\in G$.

Prove that $F_t$ is a homomorphism.

\item Prove that the homomorphism $F_t$ as described above is injective iff
$\mbox{GCD}(t,n)=1$.

Also prove that the homomorphism $F_t$ as described above is surjective iff
$\mbox{GCD}(t,n)=1$.

\item Prove that for any homomorphism $H:G\rightarrow G$ there is a
unique integer $t$ between $0$ and $n-1$ such that $H=F_t$.

Moreover, $H$ is an automorphism iff the corresponding $t$ is coprime
with $n$.

This is often stated by saying that the number of automorphisms of a
cyclic group of order $n$ is exactly $\phi(n)$ the Euler $\phi$-function
evaluated at $n$.

\item Let $G$ be any group. Prove that $F:G \rightarrow G$ defined by
$F(g)=g^{-1}$ is a
homomorphism iff $G$ is abelian. Moreover, in case it is a homomorphism,
it is actually an automorphism.

\item Let $G$ be a group two  elements $x,y$ of order $2$.
Let $n=|xy|$. If $G$ is generated by $x,y$, then prove that $G$ is isomorphic to $D_{2n}$.

{\bf Hint:} The elements $xy,x$ can be thought of as $r,s$ in the
definition of the dihedral group.

\item {\bf Challenging Problem.}
Let $G$ be a finite group and $\sigma$ an automorphism of $G$.
Assume that $\sigma^2 = \sigma \circ \sigma = Id$.

Also assume that $\sigma(x)=x$ iff $x=e$.

Prove that $G$ is abelian!

{\bf Hint:}
Try the following steps.
\begin{enumerate}
\item Consider the map $x\rightarrow x^{-1}\sigma(x)$ from $G$ to $G$.
Using the given conditions prove that this map is injective.

Since the group $G$ is finite, every  element $y$ of the group can be
written as $y=x^{-1}\sigma(x)$ , i.e. $\sigma(x)=xy$ for some $x$.

\item Using $\sigma$ deduce that $y\sigma(y)=e$.
\item Use known results to finish!


\end{enumerate}

\end{enumerate}

\subsection{Group Actions.}

If $G$ is a permutation group contained in some $S_n$, then every
elements of the group permutes elements of the set $\{1,2,\cdots,n\}$.
We say that the group $G$ acts on the set $\{1,2,\cdots,n\}$.

We now generalize this concept by letting an abstract group act on a
set.

We make a

\deff{Group acting on a set}\index{group!acting on a set}
Let $G$ be a group and let $A$ be some set.
Suppose that we have a map $T: G\times A \rightarrow A$ such that
for every fixed $g\in G$ the map $a \rightarrow T(g,a)$ is a permutation
of the set $A$.

We also require the following compatibility
conditions:
\begin{enumerate}
\item
$$T(g_1,T(g_2,a))=T(g_1g_2,a)~~ \forall g_1,g_2\in G \mbox{ and } a
\in A.$$
\item $$T(e,a) = a ~~\forall a \in A$$ where $e=e_G$ the identity of
$G$.
\end{enumerate}

Then we say that $T$ is a {\bf group action }\index{group!group action}
and that $G$ acts on $A$ by the action $T$.

As before, we shorten the notation, letting go of the symbol $T$ in
favor of a single dot, so instead of $T(g,a)$ we agree to write $g\cdot
a$.

{\bf Warning:} It is important not to confuse the ``$\cdot$'' with the
group operation.

\subsubsection{Properties of group actions.}

Here are a few observations about the group actions.
\begin{enumerate}
\item Let $T$ be a group action of a group $G$ on a set $A$.
Let the permutation defined by $a\rightarrow T(g,a)$ be denoted by
$\sigma_g$. Then it is easy to see that the map $\sigma:G \rightarrow
S_A$ given by $g \rightarrow \sigma_g$ is a group homomorphism! Indeed,
the compatibility conditions are designed to exactly guarantee this.

Thus, we could have simply defined  a group action of $G$ on $A$ as a
homomorphism of $G$ in to $S_A$.

\item It is possible to have a trivial action, defined as $T(g,a)=a$
for all $g\in G$ and $a\in A$. It is not very interesting, of course.
\index{group!action!trivial} 

\item It is more interesting to consider actions for which $\sigma_g \neq Id$
unless $g=e$. Such actions are said to be {\bf faithful.}
In this case the kernel of the homomorphism $g \rightarrow \sigma_g$ reduces to
identity $\{e\}$ and the homomorphism is injective.
\index{group!action!faithful}


\item One of the simplest actions is obtained when we let a group act on
itself by left multiplication. Thus, we define $T(g,a) = ga$ for all
$g,a\in G$.

Note that this action is always faithful and it is called {\bf the left
regular action.} We can obviously define a corresponding {\bf right
regular action}
by taking $T(g,a)=ag$.
\index{group!action!regular}


When the action is faithful, the homomorphism $g\rightarrow \sigma_g$
is indeed an isomorphism between $G$ and its image. Thus the group can
be thought of as a subgroup of the permutation group $S_A$ up to
isomorphism.

Thus, we have proved the famous {\bf Cayley Theorem} Every group is
isomorphic to a subgroup of a permutation group.
\index{Theorem!Cayley}

\item Given a finite group $G$ with $n>1$ elements, we can think about
the smallest $d$ for which $G$ has an isomorphic copy in some $S_d$.

Cayley's Theorem shows that the smallest $d$ is less than or equal to
$n$. There are examples of groups of arbitrary order
for which the smallest $d$ is equal to $n$. (Can you find them?)

Thus, in some sense, the regular representation is the best possible.
However, for many groups, we can make isomorphic copies in smaller
permutation groups, which helps in their study.

We introduce some new concepts to help us develop a few more interesting
useful actions.

\end{enumerate}

\subsection{Cosets of a subgroup.}
Let $H$ be a subgroup of a group $G$. We make a
\deff{Coset}\index{group!coset}
For each $g\in G$ the set $gH = \{ gh | h\in H\}$ is defined to be
the left coset of $H$ and is called the left coset of $H$ defined by
$g$.


It is easy to define a right coset $Hg$ similarly and it is shown to
have similar properties. We leave the parallel treatment for the reader
to finish.

It is however {\bf important to remember} that in general as sets $gH \neq Hg$
and the condition of when they are equal has important theoretical
consequences.

We easily see that $g_1H=g_2H$ iff $g_1^{-1}g_2 \in H$.
Moreover
$$z\in g_1H\bigcap g_2H \mbox{ iff } z = g_1h_1=g_2h_2 $$
for some $h_1,h_2\in H$.
Thus we see that $z\in g_1H\bigcap g_2H$ implies that $g_1^{-1}g_2\in
H$.

Thus, we have proved that two cosets $g_1H,g_2H$ are either disjoint or
equal.

{\bf For technical convenience} Assume that $G$ is a finite group.
This guarantees that we have only finitely many left cosets of $H$.
We can handle the case of infinite groups as well, but some adjustments
are needed. These will be taken up later.

Choose elements $g_1=e,g_2,\cdots,g_r, \in G$ such that $A=\{g_iH |
i=1,\cdots,r \}$ is the set of all possible distinct left cosets of $H$ in $G$.

It is then obvious that $G$ is the disjoint union of all the left cosets
of $H$ and moreover any two cosets obviously have the same number of
elements, namely $|H|$.

Thus, we see that $|G|=r|H|$ and we have proved the famous

{\bf Lagrange Theorem }\index{Theorem!Lagrange}
For any finite group $G$ of finite order $n$, the order $|H|$ of any
subgroup $H$ divides $n$. In particular, the order $|g|$ of any
element of $G$ also divides $n$, since we already know that $|g|$ equals the order
of the cyclic subgroup $<g>$.

The number $r=\frac{|G|}{|H|}$ is clearly useful and is called the {\bf
index} of $H$ in $G$. We shall use the notation $[G:H]$\index{group!index of a subgroup}

Now we describe the {\bf action of a group on its cosets.}
\footnote{This action will become a very important tool later on. Be
sure to study many examples.}


Define an action $T$ of $G$ on $A$, the left cosets of $H$,  as follows:

For a given $g\in G$ and $g_iH\in A$ we know that the coset $(gg_i)H$ is
a member of $A$ and hence is equal to $g_jH$ for some $j$.
Define $T(g,g_iH)=g_jH$.

For convenience, let us denote the map from $G$ into $S_{A}$ by the
symbol $\psi$ and let us agree to write the permutation $\psi(g)$ by
setting $\sigma_g(g_iH)=g_jH$ if $gg_iH=g_jH$. Thus, $\sigma_g$ is
the name of the permutation function corresponding to the element
$\psi(g)\in S_A$.
\footnote{In some sense, we don't need a new symbol, we could have
simply written, $\psi(g)(g_iH)=gg_iH=g_jH$. If you find it more
convenient, you may skip the notation $\sigma_g$.}

This gives a homomorphism of $G$ into $S_r$, where $r$ is the index $[G:H]$.
Is this faithful?

The kernel is the set of those elements $h\in G$ such that $hgH=gH$ for
all $g\in G$. What does this mean?

We see that
$$hgH=gH  ~~\mbox{\bf iff } ~~g^{-1}hgH = H  
 ~~ \mbox{\bf  iff   } ~~ g^{-1}hg \in H .$$
Since the condition $g^{-1}hg \in H$ holds for all $g$, applying it with
$g=e$ we see that $h\in H$.

Indeed the kernel of the homomorphism is a subgroup $K$ of $H$ such that
all conjugates of $K$ are again contained in $H$ and it is the largest
such subgroup!(This needs a little thought.) It is often called {\bf the
core} of $H$ in $G$.\index{Core of a subgroup}

\subsection{Applications of the above theory.}
\begin{enumerate}
\item {\bf Groups of prime order} Let $G$ be a group of order $p>1$, a
prime number. Then $G$ is a cyclic group isomorphic to $\DZ_p$.

{\bf PROOF.}
Let $x\in G$ such that $x\neq e$. Then $|x|>1$ and divides the prime
$p$, so $|x|=p$. Thus the $<x>\subset G$ both have the same number of elements
and hence $G=<x>$.

Define a map $\psi:\DZ_p \rightarrow G$ by $\psi(\overline{a})=x^a ~~~~~
\forall a \in \DZ$.

{\it Note that we defined the map by choosing $a\in \DZ$ and describing
$\psi(\overline{a})$. This has the advantage that it is easy to write,
but we need to show that it is well defined. For this, we write:}

We check that $\psi$ is well defined, i.e. if $\overline{a} =
\overline{b} $ then $\psi(\overline{a}) = (\overline{b})$ i.e. 
$x^a = x^b$.

Now we know that: 
$$ \overline{a}=\overline{b} \mbox{ iff } a = b+mp \mbox{ for some } m
\in \DZ.$$
Also we know that:
$$ x^a=x^b \mbox{ iff } a = b+mp \mbox{ for some } m
\in \DZ.$$

Hence $\psi$ is well defined. The homomorphism property is easy. The two
observations above also show that the map is injective. Hence
$|\psi(\DZ_p)|=p=|G|$. Therefore the map is also surjective, thus $\psi$
is an isomorphism from $\DZ_p$ to $G$. 

This is often paraphrased by saying that for a given prime $p$,  
up to isomorphism, there is only one group of order $p$. 

{\bf Moreover, note that } we never used the primeness while proving the
properties of $\psi$. Hence, the above argument also shows that  any
cyclic group $G$ of order $n$ is isomorphic to $\DZ_n$. 

\item {\bf Analyzing the dihedral groups.}
Recall that a dihedral group $D_{2n}$ was described  as a group with
generators $r,s$ such that $|r|=n,|s|=2$ and $r^s = r^{-1}$. Since
$s^2=e$ we see that $r^s = srs^{-1} = srs = r^{-1}$.


We shall now study the group action on cosets for the example of a
dihedral group.

\begin{itemize}
\item Let $G=D_{2n}$ and $H=<r>$. As usual, assume $n\geq 3$. 
Note that $|H|=n$ and hence $|G|=2|H|$.

Thus we have only two cosets, $eH = H$ and $sH$. For any $r^i\in H$ we
see that $T(r^i,H) = r^iH = H$ and using $s^2=e$ as well as $srs =
r^{-1}$ we see that: 
$$T(r^i,sH) = r^isH = ssr^isH = sr^{-i}H =  sH.$$

Thus each $r^i\in Ker(T)$, i.e. induces the identity permutation on
the pair $(H,sH)$.

The element $s$ in turn, produces the swap of the two cosets, namely
$T(s,H)=sH$ and $T(s,sH)=ssH = H$. Further, it now obvious that 
$T(sr^i,H) = T(s,T(r^i,H))= sH$ and $T(sr^i,sH) = T(s,T(r^i,sH))=H$.

\item Now take $G=D_{2n}$ with $n\geq 3$ as above, but take the subgroup to be $K=<s>$.
Then we have $n$ cosets $K,rK,\cdots,r^{n-1}K$. For convenience, name
them as $K_i = r^{i-1}K$.

Consider the corresponding action $T(g,K_i)=gr^{i-1}K$.

First, we look for the kernel. By theory, it is a subgroup of $K$, so we
only need to check $T(s,K_i)= sr^{i-1}K$. But 
$$sr^{i-1}K=sr^{i-1}ssK = r^{1-i}sK = r^jK$$
where $j$ is the remainder of $1-i$ modulo $n$.
Already, for $i=2$ we get $T(s,K_2) = T(s,rK) = r^{-1}K = r^{n-1}K =
K_n$. Since $n\geq 3$, we get that this is not the identity permutation
and hence the kernel $Ker(\psi)=\{e\}$.

Thus, $G$ is isomorphic to its copy in the permutation group
$S_{\{K_1,\cdots,K_n\}}$.
We shall explicitly calculate the image by determining the permutations
$\sigma_r,\sigma_s$. For convenience and understanding, we shall write
these as permutations of $(1,2,\cdots,n)$, i.e. elements of $S_n$ rather
that $S_{\{K_1,\cdots,K_n\}}$.

Note that $\sigma_r(K_i)=rr^{i-1}K = r^iK$. Thus as a member of $S_n$ we
get $\sigma_r = \matr{4}{1 & 2 & \cdots & n}$.

Now, $\sigma_s(K_i) = sr^{i-1}K =r^{1-i}sK = r^{n+1-i}K=K_{n-i}$.
Thus $\sigma_s(K_1) = K_{n-1}$ and $\sigma_s(K_{n-1})=K_1$. 
Thus clearly 
$$\sigma_s = \matr{2}{1 & (n-1)} \matr{2}{2 & (n-2)} \cdots \matr{2}{i & (n-i)}
\cdots .$$

Note that the middle two cycle may collapse.

For example, for $n=3$, we get $\matr{2}{1 & 3}\matr{2}{2 & 2}
= \matr{2}{1 & 3}$.

For $n=4$ on the other hand, we get $\matr{2}{1 & 4} \matr{2}{2 & 3}$.

The resulting isomorphic copy of $G$ is generated by the two elements 
$\sigma_r$ and $\sigma_s$.

For concrete values of $n$, the permutation representation is easier to
calculate with.


\end{itemize}
\item {\bf Existence of a normal group.}
Let $G$ be a group of order $15$. Assume that $G$ has an element $x$ of
order $5$ and $y$ an element of order $3$.\footnote{Both of these 
assumptions will be known facts after some more theory. 
See Cauchy Theorem proved later.}  

Here is a dramatic conclusion.  

We claim that the group
$H=<x>$ is necessarily a normal subgroup of $G$.

{\bf Proof.} Consider the cosets of $H$ in $G$. We must have exactly $3=
[G:H]=\frac{|G|}{|H|}$.

We claim that $y\not\in H$, since otherwise the group $H$ of order $5$
will contain a subgroup $<y>$ of order $3$ which is not a factor of $5$.

Thus $H \neq yH$. By a similar reasoning using $<y^2>=<y>$, we see that
the three cosets are $A=\{H,yH,y^2H\}$.

Thus the coset action map gives $$\psi:G \rightarrow S_{A}
\cong S_3.$$

Now $\psi(G)$ the image of $G$, is a subgroup of a group of order $6$ 
(namely $S_A$.) Hence $\psi$ cannot be injective. (Otherwise, the order
of $\psi(G)$ would be also $15$ and cannot divide $6$.)

The $Ker(\psi)\neq \{e\}$ being a subgroup of $H$, must be a group of order $5$
(since the order divides $5=|H|$) and hence $Ker(\psi)=H$.

As we already know, kernel of any homomorphism is a normal subgroup,
hence our claim is proved.

\item {\bf We can indeed go on to prove} that the group $G$ of order $15$ 
is actually abelian and in
fact, a cyclic group generated by $xy$. 

We shall then consider a challenge to extend this to groups 
of order $pq$ where $1<p<q$ are prime numbers.

This uses other ideas already
developed and goes thus:

\begin{enumerate}
\item First we show that $G$ is abelian:

Suppose, if possible, that $G$ is not abelian.
Note that then $x,y$ don't commute, since otherwise, it is easy to see
that all elements of $G$ commute with each other.

Now that we know that $H \vartriangleleft G$, we know that
$yxy^{-1}=x^i\in H$ for some $i=1,2,3,4,5$. Define the map $\theta:H
\rightarrow H$ by $\theta(g)=g^y$. 

Since $x,y$ don't commute by assumption, we see that $i\neq 1$ and that 
$\theta$ is an
automorphism of $H$. Thus $i$ is one of $i=2,3,4$.

Now $\theta^2(x)=\theta(x^i) = x^{i^2}$ and $\theta^3(x)=x^{i^3}$. 
But $\theta^3(x) = y^3xy^{-3}=x$. Since $y^3=e$, we must have $i^3 \equiv 1
\pmod{5}$. By checking $i=2,3,4$, we see a contradiction in each case!

\item Now we have shown $G$ to be abelian and $|x|=5,|y|=3$.
The intersection of the two subgroups $<x>$ and $<y>$ must be $\{e\}$
since its order must divide the coprime numbers $3,5$ (by Langrange
Theorem).

We claim that $|xy|=\mbox{LCM}(3,5)=15$.
First note that $(xy)^{15} = x^{15}y^{15} = e$, so $|xy| | 15$.

Suppose $|xy|=n$.

Now $(xy)^n=e$ implies $x^ny^n=e$ and this means 
$$x^n = y^{-n} \in
<x>\bigcap <y> = <e>.$$

It follows that $x^n=y^{-n}=e$ and hence $\mbox{LCM}(3,5)|n$.

Thus $|xy|=n=15$ and the cyclic group $<xy>$ must coincide with the
group $G$ having the same order.

\item {\bf Challenge.}
The above argument can be repeated for several possible cases as
follows.

Let $G$ be a group of order $pq$ where $1<p<q$ are primes.
As we shall show later, by the Cauchy Theorem,  $G$ necessarily has elements of orders $p,q$
called $y,x$ respectively.
 
{\bf Find out conditions on primes $\mathbf{p,q}$} for which you can prove that $G$ 
must be a cyclic group of order $pq$.

{\bf Hint:} Note that for $p=2$, you always have the example of the
dihedral group, so you cannot allow $p=2$.

Also, for $p=3$ and $q=7$, it is possible to have a non abelian group
defined by $y^3=x^7=e,x^y = x^{2}$. Check it out and understand why it
escapes our method of proof above.


\end{enumerate}
\end{enumerate}

\section{Preliminary Theorems in Groups.}
\subsection{Normalizers, centralizers and
stabilizers.}\index{group!normalizer} \index{group!centralizer}
\index{group!action!stabilizer} \index{group!action!orbit}

Suppose that a group $G$ acts on a set $A$ and let $\sigma_g$ denote the
permutation of $A$ defined by $a \rightarrow g\cdot a$.


We {\bf define} the orbit of a subset $B\subset A$ as follows:

The orbit of $B$ under the action of $G$ is 
$$G\cdot B = \{ g \cdot b | g \in G, b\in B\}.$$

We are using our convenient notation $G\cdot B$.

As before, if $B$ is a singleton $\{b\}$, then we agree to shorten the
notation  $G\cdot \{b\}$ to $G\cdot b$.

We now define two important concepts associated with a subset of $A$.

{\bf We define the stabilizer of a set}  $B\subset A$ to be
$$G_B = \{g \in G | g\cdot B \subset B\}.$$

It is easy to show that $G_B$ is indeed a subgroup of $G$ and is the 
largest subgroup which induces an action on the subset $B$.

When $B=\{b\}$ for some element $b\in A$, we may simplify the notation
$G_{\{b\}}$ to $G_b$.

Given a subset $B\subset A$, we can consider another smaller subgroup
of $G$, namely 
$$\{g\in G | g\cdot b = b \ \ \  \forall b \in B\}.$$
This is easily seen to be the intersection $\bigcap\{G_b | b \in B\}$.
 We can call it {\bf the fixer of 
the subset $\mathbf{B\subset A}$}.\index{group!action!fixer}

We may use more suggestive notations as follows:
$$\mbox{\bf Stabilizer     } Stab(B;G) = G_B$$
and 
$$\mbox{\bf Fixer     } Fix(B;G) = \bigcap\{G_b| b \in B \}.$$

Note that in general,  the fixer is a smaller subgroup than the stabilizer.

Given any two elements $a,b\in A$ it is easy to see that the orbits 
$G\cdot a$ and
$G\cdot b$ are either equal or disjoint. 

To see this, first note that for any $a\in A$, 
 $G\cdot (g\cdot a) = (Gg)\cdot a = G\cdot
a$. 

Thus, if $G\cdot a$ and $G\cdot b $ have a common element $c\in A$,
then $c=g_1\cdot a = g_2\cdot b $ for some $g_1,g_2 \in G$.
Hence $G\cdot c= G\cdot a$.
Similarly, $G\cdot c = G\cdot b$ and hence our claim is proved.

We now make a series of simple calculations leading to a theorem.
\begin{description}
\item[Orbit length]

The number of elements in the orbit containing $a$
is exactly $[G:G_a]$.

{\bf PROOF.} Consider the left cosets of $G_a$. First observe that 
for any $G_a\cdot a = \{a\}.$ Now we see that for any coset
$gG_a$ we get $gG_a\cdot a = \{g\cdot a\}$. Also, then for different
elements $g_1\cdot a, g_2\cdot a$ in the orbit of $a$ the corresponding
cosets $g_1G_a$ and $g_2G_a$ must be distinct, since their actions on
$a$ give different elements $g_1\cdot a$ and $g_2\cdot a$.


\item[Orbit Equation] Let us choose a collection of $m$ elements of $A$, say $\{a_i\}$ such
that every orbit contains exactly one  $a_i$. Then we have the obvious  
equation:\index{Equation!Orbit equation}
$$|A| = \sum_i |G\cdot a_i| = \sum_i[G:G_{a_i}].$$

This equation is seen by writing $A$ as a disjoint union of orbits and
adding up the number of elements in each. 

Now we apply the above equation to various special actions and
derive useful theorems.

\item {\bf Case of $\mathbf{\DZ_p}$  acting on a subset of  $\mathbf{G^p}$}

Let $G$ be a finite group and $p$ a prime factor of $|G|$. 
Let $A$ be the subset of $G^p$ defined as follows:

$$A = \{ (g_1,g_2,\cdots,g_p) | g_i \in G \mbox{ and } g_1g_2\cdots g_p =e\}.$$

Let the group  $\DZ_p$ act on $A$ by the action 
$$\overline{1}\cdot (g_1,g_2,\cdots,g_p) = (g_2,g_3,\cdots,g_p,g_1).$$
In other words, $\overline{1}$ rotates the sequence cyclically to the
left. In general, $\overline{a}$ acts by rotating $a$ times to the left. 

To check the required properties, it is necessary to note that for for
any $p$-tuple $g=(g_1,g_2,\cdots,g_p)\in A$ the resulting $p$-tuple is
also in $A$, i.e. $g_2g_3\cdots g_pg_1$ is also equal to $e$.

For any such $g\in A$, we can also see that the stabilizer $(\DZ_p)_g$
is either $\{\overline{0}\}$, if at least two of the $g_i$ are different and
equal to $\DZ_p$ in case all components of $g$ are the same as $g_1$
and thus $g_1^p=e$.

Thus the orbit of a $g$ with at least two different components has
exactly $p$ elements (i.e. $[\DZ_p:<\overline{0}>]$) or exactly $1$
element when $g=(g_1,g_1,\cdots,g_1)$.

It is easy to count that $|A|=|G|^{p-1}$ since the $p$-tuples of $A$ can
be thought of as random $(p-1)$ elements of $G$ followed by the inverse
of their product as the last element.

Thus we have an equation 
$$|G|^{p-1} = \mbox{ number of singleton orbits} + p(\mbox{ number of
distinct orbits of length $p$}).$$

Since the left hand side of this equation as well as the second part of
the right hand side are both divisible $p$, the number of singleton
orbits is also a multiple of $p$. One singleton orbit is
$\{(e,e,\cdots,e)\}$ and so there must exist an element $x\neq e $ in
$G$ such that $(x,x,\cdots,x)\in A$, i.e. $x\neq e$ but $x^p=e$.

Thus we have proved the

{\bf Cauchy Theorem.} If a prime $p$ divides $|G|$, then $G$ has an
element $x$ of order $p$.\index{Theorem!Cauchy}
This is actually the elegant {\bf proof by McKay} of the original Cauchy
Theorem.\index{Theorem!Cauchy!Proof by McKay}

The original proof is an intricate induction proof with a special
argument for the abelian case. It is a good exercise in induction!

{\bf Food for thought.} The proof shows that indeed there are at least
$p-1$ elements of order $p$. Do you see how to get the others?

Where did we use the fact that $p$ was a prime? Could the proof still
work if $p$ is an arbitrary integer? Should the theorem be even expected
to be true?

\item[The Class Equation]
An important group action happens when a group $G$ acts on itself by
conjugation. Thus here for elements $g,h\in G$ we let  $g\cdot h$ be 
$h^g = ghg^{-1}$.

It is well worth redefining and renaming the various subgroups mentioned
above.

For $h\in G$ the stabilizer is defined as 
$$C_G(h) = \{ g\in G | ghg^{-1}=h\}.$$
Its elements commute with $h$ and the group is called {\bf the centralizer} of $h$. 

The fixer of any subset $S\subset G$ is also called its centralizer  
$$C_G(S) = \{ g\in G | gsg^{-1}=s \ \ \ \forall s\in S\}.$$

On the other hand the stabilizer of the subset $S$ is denoted by
$$N_G(S) = \{ g\in G | gSg^{-1}\subset S\}.$$
This is called {\bf the normalizer} of $S$.

There is a reason for the term normalizer. If $S$ is a subgroup of $G$
then the normalizer $N_G(S)$ is a group between $S$ and $G$ with the
property that $S$ is a normal subgroup of $N_G(S)$. Moreover, it can be
shown to be the largest such group.

Under the conjugation action, the orbit of an element $h$ is the set of
all possible conjugates $ghg^{-1}$  of $h$ and the number of such
conjugates is the index of its centralizer $[G:C_G(h)]$. A singleton
orbit clearly consists of an element which commutes with every element
of the group. Such elements are said to be central. 
In turn, it is possible to prove that all central elements form a subgroup of $G$ 
called its {\bf center} and this group is denoted by the symbol
$$Z(G) = \{ g\in G | ghg^{-1}=e \mbox{ i.e. } gh=hg \ \ \ \forall h\in
G\}.$$
\index{group!center of}

The orbit equation now takes the following shape:

$$|G| = \mbox{ number of singleton orbits} + \sum [G:C_G(h)]$$
where the sum is taken by choosing one representative from each
conjugation orbit with at least two elements. In view of the definition
of the center, we get the famous {\bf Class
Equation:}\index{Equation!Class Equation}

$$|G| = |Z(G)| + \sum [G:C_G(h)]$$
the sum being over distinct non singleton orbits as before.

\item {\bf A property of a $\mathbf{p}$-group.}\index{group!p-group}
Let $p$ be a prime.
We define a group $P$ to be a {\bf $\mathbf{p}$-group} if $|P|$ 
is a power of $p$.
\footnote{We can alternatively define a $p$-group as a group where the
order of every element is a power of the prime $p$. Then we need to use
the Cauchy Theorem to show that this definition is equivalent to the
above for a finite group. This definition has the advantage that it
makes sense even for infinite groups!}

{\bf Theorem on $\mathbf{p}$-groups.}
If $p$ is a prime and $P$ is a $p$-group, then the center $Z(P)$ is non
trivial, i.e. has at least one non identity element.

Let $P$ act on itself by conjugation and notice that all the indices
appearing as lengths of non singleton orbits are powers of $p$, so at
least divisible by $p$.

Since the left hand side $|P|$ of the class equation is also divisible by $p$,
we get that $p$ divides $|Z(P)|$. Since $e\in Z(P)$ we have $|Z(P)|\neq
0$, hence it contains non identity elements!


\end{description}

\subsection{Quotient groups.}
Before we proceed further it is useful to understand how the
homomorphisms and especially their kernels arise.

We have already shown that the kernel of any homomorphism is a normal
subgroup of the domain. We shall now show the converse that a normal
subgroup of a group is always the kernel of a certain natural
homomorphism.

{\bf Existence of a quotient group}\index{group!quotient group}
Let $G$ be any group and $H$ a subgroup. Let $A$ be the collection of
all left cosets of $H$ in $G$.

Assume that $H$ is a {\it normal} subgroup of $G$. Then $A$ is naturally
a group such that
\begin{enumerate}
\item For any $g_1H,g_2H$ we define $g_1Hg_2H=g_1g_2H$.
\item There is a natural homomorphism $\Psi:G\rightarrow A$ defined by
$\psi(g)=gH$.
\item The homomorphism $\Psi$ is surjective and its kernel is exactly
$H$.
\end{enumerate}

{\bf Proof.}
Here are the steps of the proof.
\begin{enumerate}
\item Note that the normality of $H$ implies $gH=Hg$ as sets for any
$g\in G$. Thus the definition $g_1Hg_2H=g_1g_2H$ is simply a calculation
$$g_1Hg_2H~=~ g_1(Hg_2)H =~=~ g_1(g_2H)H ~=~ g_1g_2(HH) ~=~ g_1g_2H$$
where the last equation follows from the fact that $H$ is a subgroup and
hence $HH=H$.

\item The fact that $A$ becomes a group under this operation is left for
the reader to verify. Let  us record that the identity is simply the
coset $e_GH=H$ and we shall call it $e_A$.

Moreover, $(gH)(g^{-1}H)=(gg^{-})H=H$ implies that
$(gH)^{-1}=g^{-1}H$.

\item The fact that $\Psi$ is a homomorphism is evident from its
definition.

\item The map $\Psi$ is surjective since any $gH\in A$ is equal to
$\Psi(g)$ by definition. Also $\Psi(g)=e_A$ iff $gH=H$ iff $g\in H$.
hence $Ker(\Psi)=H$.

\end{enumerate}

{\bf Notation.} We shall conveniently write $G/H$ for the set of cosets
$A$ and call it the quotient group of $G$ by $H$.

{\bf Warning:} It is important to remember that the quotient group $G/H$
is defined only when $H$ is a normal subgroup, but the cosets as a set
are defined for any subgroup. We don't get the group structure if $H$ is
not normal.

\subsubsection{Generation of groups.}
Let $G$ be a group and $H,K$ its subgroups.
We consider the set $HK = \{hk|h\in H,k\in K\}$. We are interested in
deciding if it a subgroup and if indeed it is equal to $G$ itself.

Here is a sequence of useful results.
\begin{enumerate}
\item $HK$ is a subgroup of $G$ iff $HK=KH$.

{\bf PROOF.} We use our convenient subgroup criterion, namely $L\subset G$ 
is a subgroup iff $LL=L$ and $L^{-1}=L$.

Suppose $HK=KH$. Then $HKHK=H(KH)K = HHKK$ and hence reduces to $HK$
since the fact that $H,K$ are subgroups says that $HH=H$ and $KK=K$.

Also, $(HK)^{-1} = K^{-1}H^{-1}$ reduces to $KH$ since $H,K$ are
subgroups and hence to $HK$ by hypothesis.

Thus $HK<G$.

Now assume that $HK<G$. 
Then $(HK)^{-1}=K^{-1}H^{-1} = KH$ since $H$ and $K$ are subgroups. But
$(HK)^{-1}=HK$ since $HK$ is a subgroup, thus we have proved $HK=KH$.

\item If $H,K$ are subgroups of $G$ and if $H<N_G(K)$ then $HK<G$.
Similarly, if $K<N_G(H)$ then we also get $HK<G$.

{\bf PROOF.} 
If $H<N_G(K)$ then $hK=Kh$ for each $h\in H$ and hence $HK=KH$.
Therefore we are done by the above result.

Similar proof holds if $K<N_G(H)$.

In view of this result, we make a 

\deff{A group $\mathbf{H}$ normalizes a group $\mathbf{K}$} (in
$\mathbf{G}$) if $H<N_G(K)$.

Similarly, we make a 

\deff{A group $\mathbf{H}$ centralizes a group $\mathbf{K}$} (in
$\mathbf{G}$) if $H<C_G(K)$.
\index{group!normalizing subgroup}
\index{group!centralizing subgroup}


\item {\bf Corollary.} If $H,K$ are subgroups of $G$ and either $H$ or
$K$ is normal then $HK$ is a subgroup of $G$. 

{\bf PROOF.} The hypothesis means either $N_G(H)$ or $N_G(K)$ equals $G$ 
and hence we are done by the above result.

\item {\bf Counting $\mathbf{HK}$.}
If $H,K$ are finite subgroups of $G$ then we can count the elements of the set
$HK$ regardless of whether it is a subgroup and this is sometimes
useful.

We can regard $HK$ as a union of left cosets of $K$ by elements of $H$.

Let $L$ be the subgroup $H\bigcap K$ (This is easily seen to be a
subgroup).

Note that $L$ is a subgroup of both $H,K$. Using this, write 
$$K=\bigcup_{i=1}^m k_iL \mbox{ where }  m=|K|/|L| = [K:L].$$

Similarly, write 
$$H=\bigcup_{j=1}^n h_jL \mbox{ where }  n=|H|/|L| = [H:L].$$

Then we see that
\footnote{ First write $HK=\bigcup h_jK$ noticing that these are exactly
all the distinct cosets of this form.
Then write $K = \bigcup k_iL$ in the usual coset representation.}

$$HK = \bigcup_{j=1}^n \bigcup_{i=1}^m\left(  h_jk_iL \right)$$
and hence 
$$|HK|=nm|L| = \frac{|H||K||L|}{|L||L|} = \frac{|H||K|}{|L|}.$$

Thus we have proved:
$$|HK| = \frac{|H||K|}{|H\bigcap K|}.$$

\end{enumerate}

\subsubsection{Isomorphism Theorems.}
{\bf Preamble.}

If $G$ is a group and $\phi:G\rightarrow L$ is a homomorphism, then we
know that $\phi(G)<L$ and $Ker(\phi)<G$. We also know that $Ker(\phi)$ 
is in fact a normal subgroup, say $H\triangleleft G$. Thus the quotient 
group $G/H$ is defined as the group formed by the left cosets of $H$ in
$G$. 

The first important theorem states that 

{\bf First Isomorphism Theorem.}
With the preamble in force, we have $G/H \cong \phi(G)$.

{\bf PROOF.}
Define a map $\theta: G/H \rightarrow \phi(G)$  by $\theta(gH) =
\phi(g)$.

We note that if $g_1H=g_2H$ then $g_1=g_2h$ for some $h\in H=Ker(\phi)$
and hence $\phi(g_1) = \phi(g_2h) = \phi(g_2)\phi(h) = \phi(g_2)$ since 
$\phi(h)=e_L$.

This shows that $\theta$ is well defined. The fact that it is a
homomorphism is now easily checked.

Also, $\theta(gH)=e_L$ iff $g\in Ker(\phi) = H$ iff $gH=H$.
Thus $Ker(\theta) = \{H\}$.

Thus $\theta$ is injective. It is clearly surjective onto
$\phi(G)$ proving the result.

Next we have the

{\bf Second Isomorphism Theorem.}
Let $A,B$ be subgroups of a group $G$ such that 
$A<N_G(B)$ and thus $AB$ is a subgroup of $G$.

Then we have:
\begin{enumerate}
\item $B$ is a normal subgroup of $AB$
\item $A \bigcap B $ is a normal subgroup of $A$ and
\item We have: $$AB/B \cong A/(A\bigcap B).$$
\end{enumerate}

{\bf PROOF.}
The normality of $B$ follows since $A< N_G(B)$.
The same fact says that for $a\in A$ and $b\in B$ we have $b^a\in B$.
Moreover, if $b$ is also in $A$,  then we also have $b^a\in A$.
Thus $b^a\in A\bigcap B$ for all $b\in A\bigcap B$; this proves the
second claim.

Finally, let $F:A \rightarrow AB/B$ be the map defined by $F(a)=aB$.
We show that $F$ is a surjective homomorphism with kernel $A\bigcap B$,
thereby proving the result from the first isomorphism theorem.

First $F(a_1a_2)=a_1a_2B = a_1Ba_2B$ since $B$ is normal in $AB$.
Thus $F$ is a homomorphism.

For any $abB\in AB/B$ we see that $abB = aB$ since $b\in B$ and thus $F$
is surjective.

Now for $a\in A$, we see that $F(a)=B$ iff $aB=B$ i.e. $a\in B$. Thus
the kernel is exactly $A\bigcap B$.

Hence, we are done by the first isomorphism theorem.

We now derive a very useful {\bf Homomorphism Principle.}

Assume that we have a group $G$ with a normal subgroup $H$ and some
group $K$.

Then any homomorphism $F$ from $G/H$ to $K$ induces a homomorphism $F^*$
from $G$ to $K$ such that $F^*(H)=\{e_K\}$. This is defined as
$F^*(g)=F(gH)$.

Conversely, given any homomorphism $F^*$ from $G$ to $K$ satisfying
$F^*(H)=\{e_K\}$ we can find a homomorphism $F$ from $G/H$ to $K$ defined
by $F(gH)=F^*(g)$.

We describe this by saying that $F$  is a factor of $F^*$.

Typically, homomorphisms from a quotient group $G/H$ are created as
factors of convenient homomorphisms of $G$.

{\bf PROOF.} The first part is obvious by the definition of the quotient
group. The second part only needs the verification that the map is well
defined, i.e. to check that $g_1H=g_2H$ implies $F(g_1H)=F(g_2H)$, i.e.
$F^*(g_1)=F^*(g_2)$, but this follows since $g_1=g_2h$ for some $h\in H$
and hence $F^*(g_1)=F^*(g_2)F^*(h)=F^*(g_2)$ by hypothesis.

{\bf Third Isomorphism Theorem.}
Suppose we have a tower of groups $H< K< G$. Further assume
that $H,K$ are normal subgroups of $G$. Then it is obvious that $K/H$
is a normal subgroup of $G/H$.

Further, we have
$$(G/H)/(K/H) \cong G/K.$$

{\bf PROOF.} Define a homomorphism $\psi: G/H \rightarrow G/K$ by
$\psi(gH)=gK$.
We have the natural surjective homomorphism $\psi^*$ from $G$ to $G/K$
defined by $\psi^*(g)=gK$. Clearly $\psi^*(k)=K$ if $k\in K$ and hence by
the homomorphism principle, we get that $\psi$ factors this $\psi^*$.

Now the kernel of $\psi$ is the set of cosets of the form $kH$ where
$k\in K$, i,e, $K/H$. Thus, $K/H$ is a normal subgroup of $G/H$ and by
the first isomorphism theorem we get the result.

\subsection{Sylow Theorems.}\index{Theorem!Sylow}
Now we come to some of the most powerful theorems in group theory.
We begin with a far reaching generalization of Cauchy Theorem.

First some terminology. Given a finite group $G$ and a
prime $p$ let $p^r$ be the highest power of $p$ dividing $|G|$. Any
subgroup of $G$ with order $p^r$ is said to be a
{\bf $\mathbf{p}$-Sylow subgroup } of $G$.
The set of all such Sylow subgroups shall be denoted by $Syl_p(G)$. 
The number of distinct Sylow subgroups shall be denoted by the $n_p(G)$.
\index{group!Sylow subgroup}
We may drop the $G$ from these notations, if clear from the context.

We begin by proving that $Syl_p(G)$ is a non empty set.

{\bf First Sylow Theorem.} Given a prime $p$, amy finite group $G$
has a $p$-Sylow subgroup.

{\bf Note} that by definition, if $p$ does not divide $d$, then the
trivial subgroup $\{e_G\}$ satisfies the definition of a $p$-Sylow
subgroup and the theorem is clearly true.

{\bf PROOF.}
First we prove a

{\bf LEMMA.}
Let $B$ be a normal subgroup of a group $A$ and $C$ a subgroup $A/B$.
Let $C^*$ be the subgroup of $A$ defined as the union of the cosets
$xB\in C$. Then $|C^*|=|C||B|$.

{\bf PROOF of Lemma.}
Note that $C$ is isomorphic to $C^*/B$ and hence the result follows.

Now we prove the theorem by induction on $d=|G|$. The result being
trivial for $d=1$ we assume the result for all groups with order less
than $d$ and prove it for the given $G$.

Suppose that $p^r$ is the highest power of $p$ which divides $d$.
As already observed, it is enough to consider the case when $p$ divides
$d$ and thus $r\geq 1$. 

If the center $Z(G)$ of the group contains an element $x$ of order $p$,
then note that $<x>\triangleleft ~ G$ and $p^{r-1}$ is the highest power
of $p$ which divides $|G|/|<x>| = |G|/|x|$.

By induction hypothesis $G/<x>$ contains a $p$-Sylow subgroup $C$ of order
$p^{r-1}$ and the corresponding subgroup consisting of the union of
cosets $g<x>$ such that $g<x>\in C$ forms a subgroup of $G$ of order
$p^{r-1}p=p^r$, by the LEMMA. It gives the necessary $p$-Sylow subgroup
of $G$.

Now assume that $p$ does not divide $|Z(G)|$. Let $G$ act on itself by
conjugation and write the class equation:

$$|G| = |Z(G)| + \sum [G:C_G(h)]$$
the sum being over distinct non singleton orbits.

Since $p$ divides the left hand side of the equation and does not divide
the first term of the right hand side, it  must not divide at least one of
the terms $[G:C_G(h)]$ on the right
hand side. But then $p^r$ must divide
$|C_G(h)|=\bigfrac{|G|}{[G:C_G(h)]}$.

Clearly $C_G(h)$ is necessarily a group of smaller order that $G$, and hence
it contains a $p$-Sylow subgroup of order $p^r$ which is automatically the
needed subgroup of $G$.

Having proved the existence of $p$-Sylow subgroups, we learn to count
their number.

{\bf Sylow Theorem Part 2.} Let $p$ be a prime dividing the order of a
group $G$ and let $P\in Syl_p(G)$. Then the number of $p$-Sylow
subgroups of $G$ {\bf conjugate to} $P$ is congruent to $1$ modulo $p$.

{\bf PROOF.}
Let $S$ be the set of $p$-Sylow subgroups of $G$ conjugate to $P$.
Let $P$ act on this set by conjugation. Since $P$ is a $p$-group, it
follows that every orbit has length equal to a power $p^s$ of $p$ for
some integer $s\geq 0$.

All orbits of length bigger than $1$ will then have orbit length
divisible by $p$.

We know one orbit with length $p^0=1$, namely the orbit of $P$ itself.
If we show that there is no other singleton orbit then our orbit
equation shows that
$$|S|=1+\mbox{( sum of orbits lengths which are bigger than $1$ )} = 1+mp$$
where $m$ is some non negative integer.

Now suppose if possible, that $Q$ is another $p$-Sylow subgroup which also leads to a
singleton orbit, i.e. for all $x\in P$ we get $xQx^{-1}=Q$. This means $P$
normalizes $Q$. It follows that $PQ$ is a subgroup of $G$. Moreover,
$$|PQ| = \frac{|P||Q|}{|P\bigcap Q|}$$
would say that $|PQ|\geq |P|$. On the other hand, since every term in
the fraction is a power of $p$, it follows that $|PQ|$ is also a power
of $p$ and by the ``Sylow-Ness'' of $P$ we get $|PQ|\leq |P|$.
It follows that $PQ=P$ and then $Q<P$ and consequently $Q=P$ since
$|Q|=|P|$.
\footnote{Actually what we have established is that if $Q<G$ is any
$p$-group which is normalized by $P$, then $Q<P$.}

Now we show:

{\bf Sylow Theorem Part 3.} If $p$ is a prime dividing $|G|$, then $Syl_p(G) = S$.
Thus, any two $p$-Sylow subgroups are conjugate in $G$.

In particular $n_p(G) = |S| =  [G:N_G(P)]$ where $P$ is some $p$-Sylow
subgroup. 

{\bf PROOF.}
We already know that $S\subset Syl_p(G)$.

Suppose if possible, that $R$ is another $p$-Sylow subgroup outside $S$.
Let $R$ act on $S$. By the argument used in the above proof, we see that
this action would have no singleton orbits!

But then we get that $|S|=1+mp$ is a sum of orbit lengths of the action
by $R$, i.e. is a sum of numbers divisible by $p$. This is clearly a
contradiction!

The last part is obvious.

{\bf Sylow Theorem Complete.}\index{Theorem!Sylow}
Let $G$ be a finite group and $p$ a prime dividing $|G|$.
Then we have the following:
\begin{enumerate}
\item $Syl_p(G)$ is non empty.
\item The number $n_p(G)$ of $p$-Sylow subgroups $=|Syl_p(G)|\equiv 1 \pmod{p}$.

Moreover, $n_p(G)=[G:N_G(P)]$ for any $P\in Syl_p(G)$ and thus $n_p(G)$
is a factor of $[G:P]$ and is equal to $1$ modulo $p$.


\end{enumerate}
\subsection{Applications of the Sylow Theorems.}

\begin{enumerate}
\item Let $G$ be a finite group and let $p$ be a prime dividing $|G|$.
Prove that a $P\in Syl_p(G)$ is normal in $G$ iff $n_p(G)=1$.

\item If $G$ is a group of order $pq$ where $p<q$ are prime numbers then
$G$ is cyclic unless $q\equiv 1 \pmod{p}$.

{\bf Hint:} Let $P$ and $Q$ be the Sylow subgroups of orders $p,q$
respectively. Prove that $Q$ is a normal subgroup and $P$ is also normal
unless $q\equiv 1 \pmod{p}$.

If both $P,Q$ are normal, then prove that $G$ is abelian and indeed
cyclic.

{\bf Definitive form.} Let $G$ be a group of order $pq$ where $p<q$
are primes. Assume that $q \equiv 1 \pmod{p}$. Then either $G$ is cyclic
or $G$ is a group generated by $x,y$ with $|x|=p,|y|=q$ and
$xyx^{-1}=y^m$ for some $m$ such that $m^p\equiv 1 \pmod{q}$.

It turns out that indeed there is an integer $m$ such that  $q=1+mp$ and
this is one good choice of $m$.
More on this will be discussed later. The choice of $m$ is not unique,
but there is essentially only one choice up to automorphism.

\item Let $G$ be a group of order $pq$ where $p<q$ are prime numbers and
assume that  $q\equiv 1 \pmod{p}$.
Then either $G$ is cyclic or a group which which generated with $x,y$ of
orders $|x|=p,|y|=q$ and $xyx^{-1}=y^m$ for some $m$ such that $mp\equiv
1 \pmod{q}$.

\item Let $G$ be a group of finite order $d$. Prove that $G$ is not
simple in each of the following cases.
(i) $d=12$ (ii) $d=24$ (iii) $d=72$ (iv) $d=312$ (v) $d=351$

\item Prove that a group of order $56$ has a normal $p$-Sylow subgroup
for some $p>1$.

\item Let $n\geq 2$ be an integer and let $G$ be a group of order 
$2^n p$, where $p=2^n - 1 $ is a prime number. For example: 
$$12=2^2(2^2-1)~,~56=2^3(2^3-1)~,~992=2^5(2^5-1)$$ 

Show that $G$ is not a simple group.

\item Suppose that a finite group $G$ has $n_p(G)=5$ for some prime $p$.
Prove that $G$ has elements of order $5$ and $2$.

More generally, assume that a finite group $G$ has $n_p(G)=2^n+1$ for
some $n\geq 1$. Prove that $G$ has elements of order $2$ as well as
order $q$ for every prime factor $q$ of $2^n+1$.

\item Prove that there is no simple group of order $96$.

\item Let $G$ be a group of order $2006$. Prove that it is abelian or
has a cyclic subgroup of index $2$.

\end{enumerate}

\subsection{Simplicity of $\mathbf{A_n}$.}
\index{group!simplicity of $A_n$}
Now we discuss a large family of known simple groups. These are the
so-called alternating groups $A_n$ which are defined as follows.

Let $n\geq 2$ be an integer and let $S_n$ be the usual permutation group
acting on the set $\{1,2,\cdots,n\}$.
We have already discussed the cycle decomposition of a permutation into
disjoint cycles.
Recall that an $r$-cycle is a cycle which permutes $r$ objects in a
cyclic manner. We may call a $2$-cycle a transposition.
\index{permutation!transposition}

>From a sample calculation like
$$\matr{5}{1 & 2 & 3 & 4 & 5} =
\matr{2}{1 & 5}\matr{2}{1 & 4} \matr{2}{1 & 3} \matr{2}{1 & 2} $$
we see that any $r$-cycle can be written as a product of
$(r-1)$-transpositions.

>From a calculation like
$$\matr{2}{1 & 5} = \matr{2}{1 & 2} \matr{2}{2 & 5} \matr{2}{1 & 2}$$
we can see that the above $5$-cycle can also be rewritten as a
product of $6$ transpositions instead of the original $4$.

What is surprizing is the

{\bf Theorem of the parity of a permutation.}\index{Theorem!Parity of a
permutation}
Any permutation $\sigma \in S_n$ can be written as a product of
a finite number of transpositions. While the expression is not unique,
the parity of the number of transpositions (oddness or evenness) depends
only on the permutation.

Thus we make a

\deff{Sign of a permutation.}\index{permutation!sign of}\index{permutation!odd}
\index{permutation!even}

A permutation is said to be {\bf odd} if it can be written as a product of an
odd number of permutations and its sign is said to be $-1$.

A permutation is said to be {\bf even} if it can be written as a product of an
even number of permutations and its sign is said to be $1$.

{\bf PROOF of the Theorem of the parity.} Given a permutation $\sigma \in S_n$
and a pair of integers $i \neq j $ in $\{1,2,\cdots,n\}$ consider a
rational number
$$\frac{i-j}{(\sigma(i)-\sigma(j))}.$$

Note that interchanging the order of $i,j$ does not change the
value of this rational number.
Thus, we can define a well defined function
$$\psi(\sigma,\{i,j\})=\frac{i-j}{(\sigma(i)-\sigma(j))}.$$ 
Let $T_n$ be the set of all two element subsets of $\{1,2,\cdots,n\}$ and note
that $\psi$ is a function from $S_n\times T_n$ into $\DQ$.

Now define
$$\theta(\sigma)=\prod_{t\in T_n}\psi(\sigma,t).$$

We claim that $\theta$ is a function from $S_n$ into the multiplicative
group $\{-1,1\}\subset \DQ$ and in fact it {\bf is a group
homomorphism.}

It is easy to check that $\theta(\sigma)=-1$ if $\sigma$ is a
transposition thus $\theta(\sigma)$ is $1$ if $\sigma$ is an even
permutation and $\theta(\sigma)$ is $-1$ if $\sigma$ is an odd
permutation; thus proving the theorem.

Note that when we write out the product in $\theta(\sigma)$ the
absolute values of the terms in the numerator are $|(i-j)|$ for $\{i,j\}\in T_n$
and the same is true for the denominator. Thus, the ratio is clearly
$\pm 1$.

Also note that for convenience the members of $T_n$ can be chosen in
any convennient order. Thus we could let $i=1,\cdots,n$ and then let
$j=i+1,\cdots n$. So
$$\theta(\sigma)=\prod_{i=1}^n\prod_{j=i+1}^n \left(\psi(\sigma,\{i,j\})\right).$$

Now, given any permutation $\tau\in S_n$ we see that another way of
listing members of $T_n$ might be to consider $\{\tau(i),\tau(j)\}$ as
$i$ varies from $1,\cdots,n$ and for a fixed $i$, the index $j$ varies
from $i+1$ to $n$.

Thus we also have
$$\theta(\sigma)=
\prod_{i=1}^n\prod_{j=i+1}^n \left(\psi(\sigma,\{\tau(i),\tau(j)\})\right).$$

Then we can write
$$\theta(\sigma)\theta(\tau)=
\left(\prod_{i=1}^n\prod_{j=i+1}^n
\left(\psi(\sigma,\{\tau(i),\tau(j)\})\right)\right)
\left(\prod_{i=1}^n\prod_{j=i+1}^n
\left(\psi(\tau,\{i,j\})\right)\right).$$

A simple cancellation between the numerator of the first product with
the denominator of the second product shows that

$$\theta(\sigma)\theta(\tau)=
\prod_{i=1}^n\prod_{j=i+1}^n
\psi(\sigma\circ\tau,\{i,\j\})=\theta(\sigma\circ\tau).$$

Thus, we have proved that $\theta$ is a group homomorphism from $S_n$
into the multiplicative group $\{1,-1\}$.

The kernel of the homomorphism is then a well defined normal subgroup of
$S_n$ and we define:

\deff{Alternating group}\index{group!alternating}
$A_n$ is the set of permutations $\sigma$ in $S_n$ such that $\theta(\sigma)=1$.
Equivalently it is an even permutation. If its type is
$([r_1,s_1],\cdots,[r_m,s_m])$, then it is easy to check that
$$\theta(\sigma)=\prod_{i=1}^m\left((-1)^{(r_i-1)s_i}\right).$$

We have seen above that every permutation in $S_n$ can be written as a
product of transpositions $\matr{2}{i & j}$ where $1\leq i <j \leq n$.
We now refine this to:

{\bf Theorem on generation of $\mathbf{S_n}$ and $\mathbf{A_n}$.}
We have the following generations.
\begin{enumerate}
\item Let $n\geq 2$. Then every permutation in $S_n$ can be written as
product of permutations of the form $\matr{2}{1 & i}$ where $1<i\leq n$.

{\bf PROOF.} $$\matr{2}{i & j} = \matr{2}{1 & i}\matr{2}{1 & j}
\matr{2}{1 & i}.$$

\item Let $n\geq 2$. Then every permutation in $S_n$ can be written as
product of permutations of the form $\matr{2}{(i-1) & i}$ where $1<i\leq
n$.

{\bf Idea of the proof.}
Induct from these calculations.
$$\matr{2}{(i-2) & i} =
\matr{2}{(i-2) & (i-1)} \matr{2}{(i-1)  & i} \matr{2}{(i-2) & (i-1)}.$$
$$\matr{2}{(i-3) & i} =
\matr{2}{(i-3) & (i-2)} \matr{2}{(i-2)  & i} \matr{2}{(i-3) & (i-2)}.$$



\item Let $n\geq 2$. Then every permutation in $S_n$ can be generated with
$\matr{2}{1 & 2}$ and $\matr{4}{1 & 2 & \cdots & n}$.

{\bf Idea of the proof.}
Conjugate the first one repeatedly by the second. Then use earlier
result!

\item Let $n\geq 3$. Then every element of $A_n$ is a product of
$3$-cycles, not necessarily disjoint, of course!

{\bf PROOF.}
It is enough to show that a product of two transpositions is a product
of $3$-cycles.

Here are the calculations:

$$\matr{2}{a & b}\matr{2}{c & d} =
\matr{3}{a & c & b} \matr{3}{a & c & d}.$$

$$\matr{2}{a & b}\matr{2}{a & c} = \matr{3}{ a & c & b}.$$

{\bf Note.} This gives another proof that $A_n$ is a normal subgroup of
$S_n$, since under conjugation, a $3$-cycle always goes to a $3$-cycle!

\item Let $n\geq 5$. Any two $3$-cycles are conjugate to each other by elements
of $A_n$.

{\bf PROOF.} Let two $3$-cycles be $u=\matr{3}{x_1 & x_2 & x_3}$
and $v=\matr{3}{y_1 & y_2 & y_3}$. Make two permutations of
$\{1,2,\cdots,n\}$ as
$$x=\matr{6}{x_1 & x_2 & x_3 & x_4 & x_5 & \cdots} \mbox{ and }
y=\matr{6}{y_1 & y_2 & y_3 & y_4 & y_5 & \cdots}.$$

If the permutation $\sigma$ defined by $\sigma(x_i)=y_i$ is even then it
is the even permutation which conjugates $u$ into $v$ and our proof is
done.

If $\sigma$ is odd, set $z=\matr{6}{y_1 & y_2 & y_3 & y_5 & y_4 &
\cdots}$ and note that the permutation $\tau$ defined by $\tau(x_i)=z_i$
is then an even permutation and it also conjugates $u$ into $v$.

Thus, $u,v$ are conjugate by elements of $A_n$.

{\bf Think!} See what can go wrong if $n=2,3,4$.
\item Let $n\geq 5$.
Let $H$ be any {\it normal} subgroup of $A_n$.
If $H$ contains a $3$-cycle, then $H=A_n$.

{\bf PROOF.}
If it contains one $3$-cycle, then it contains all $3$-cycles, since
these are conjugate in $A_n$ and $H$ is normal in $A_n$. Then $H$
contains all the necessary generators of $A_n$ and we are done!


\end{enumerate}

We are now set to prove the main

{\bf Theorem. $\mathbf{A_n}$ is simple if $\mathbf{n \geq
5.}$}\index{Theorem!$A_n$ is simple if $n\geq 5$}

{\bf PROOF.} Let $n\geq 5$ and let $H<A_n$ be a normal subgroup.
By a sequence of calculations, we shall show that if $H\neq \{Id\}$, then
$H$ contains all $3$-cycles and hence $H=A_n$.

{\bf Setup.}

Thus, assume that $H$ is not $\{Id\}$ and let $T$ be the set of all
integers $r$ such that some element of $H$ has a cycle of length $r\geq
2$ in its disjoint cycle representation.

Let $s=\max\{r|r \in T\}$.

We shall show that $3\in T$ and indeed a $3$-cycle belongs to $H$.
Then we are done by the above generation results.

\begin{enumerate}
\item {\bf Main principle.} If $h\in H$ and $g\in A_n$ then
$[h,g] = h(gh^{-1}g^{-1})\in H$, since $H\triangleleft A_n$.

This is obvious since the two displayed parts are clearly both in $H$.
Typically, this is used by calculating $[h,g]$ as $(hgh^{-1})g^{-1}$
after fixing $h\in H$ and choosing convenient $g\in G$.

\item Suppose that $s\geq 4$. Then $H$ contains a $3$-cycle.

Without loss of generality we can assume that
$$\sigma = \matr{5}{1 & 2 & 3 & 4 & \cdots}\cdots \in H.$$
Set $\tau=\matr{3}{1 & 2 & 3}\in A_n.$ Then
$$[\sigma,\tau] = \matr{3}{2 & 3 & 4}\matr{3}{1 & 3 & 2} = \matr{3}{1 & 4
& 2}.$$

This is in $H$ by the main principle, hence we are done.

\item Suppose that $s=3$ then $H$ contains a $3$-cycle.

{\bf PROOF.} Without loss of generatlity we have $\sigma \in H$ of the
form $\sigma=\matr{3}{1 & 2 & 3}\cdots$. If $\sigma$ is just a
$3$-cycle, then we are done. Hence we may assume:
$$\sigma = \matr{3}{1 & 2 & 3}\matr{3}{4 & 5 &\cdots}\cdots.$$
Let $\tau = \matr{3}{1 & 2 & 4} \in A_n$.

Then we see
$$[\sigma,\tau]= \matr{3}{ 2 & 3 & 5}\matr{3}{1 & 4 & 2} = \matr{5}{1 &
4 & 3 & 5 & 2}\in H.$$

Thus $s\geq 5$ and we have a contradiction!

\item We now claim that $s\geq 3.$ 

{\bf PROOF.} Suppose if possible $s=2$. Then we can assume without loss
of generality that
$$\sigma = \matr{2}{1 & 2}\matr{2}{3 & 4}\cdots \in H.$$

Take $\tau=\matr{3}{1 & 3 & 5}\in A_n$. Here, we seriously need $n\geq
5$.
There are two possible cases.

{\bf Case 1.} $\sigma = \matr{2}{1 & 2}\matr{2}{3 & 4}$. In this case,
we see:
$$[\sigma,\tau]= \matr{3}{2 & 4 & 5}\matr{3}{1 & 5 & 3}= \matr{5}{1 & 2
& 4 & 5 & 3}\in H.$$

Thus we have $s\geq 5$. Thus $s=2$ is not possible!

{\bf Case 2.} $\sigma = \matr{2}{1 & 2}\matr{2}{3 & 4} \matr{2}{5 & 6}\cdots $.
In this case, we see:
$$[\sigma,\tau]= \matr{3}{2 & 4 & 6}\matr{3}{1 & 5 & 3}\in H.$$
But then $s\geq 3$, another contradiction!


\item {\bf Done!}

We have shown that $s\geq 3$ and for each of the possibilities
$s=3$, $s\geq 4$ we have shown that $H$ contains a $3$-cycle as promised!



\end{enumerate}
{\bf Remark.}
We now finish off the discussion by describing the situation for $n<5$.

\begin{enumerate}
\item If $n=2$ then $A_2$ is the identity subgroup and is trivially
simple.
\item If $n=3$, then $A_3$ is the cyclic group of order $3$ generated by
$\matr{3}{1 & 2 & 3}$. It is clearly simple, being of prime order.
\item If $n=4$, then $A_4$ contains a famous normal subgroup, nammely
$K$, the Klein $4$-group defined as
$$K = \{ Id,\matr{2}{1 & 2}\matr{2}{3 &4} ,
\matr{2}{1 & 3}\matr{2}{2 & 4},
\matr{2}{1 & 4}\matr{2}{2 & 3}\}.$$
\index{group!Klein $4$-group}
Note that in our above notation, here $s=2$ is possible, because we
don't have a fifth element to create our $\tau$. This $K$ is a proper
normal subgroup of $A_4$ and is abelian. Hence, every subgroup of $K$ is
normal in $K$, though not in $A_4$. The subgroup $K$ is indeed the
unique proper normal subgroup of $A_4$.

It can be shown that $K$ is in fact the commutator subgroup $[A_4,A_4]$.

\end{enumerate}

\subsection{Building new groups from old.}
Given two groups $H,K$ we now investigate various possible ways of
putting them together to form a new group. This naturally helps in
analyzing a given group as developed from smaller groups.

The first construction is the simplest product.

{\bf Direct product:}\index{group!product!direct}
\index{group!product!external direct}
Given groups $H,K$ by their {\bf direct product} we mean the set
$H\times K$ together with the group operation defined by 
$$(h_1,k_1)(h_2,k_2)=(h_1h_2,k_1k_2).$$
Note that to avoid cluttering our notation, we are not identifying the
symbols for the different group operations and thus $h_1h_2$ is defined
as the product in $H$ while $k_1k_2$ is defined as the product in $K$.

This direct product is also called the external direct product in
contrast with the internal direct product defined below.

We note the following easy facts about the direct product.

\begin{enumerate}
\item If $H,K$ are finite then $|H\times K| = |H||K|$.
\item If $L$ is a subgroup of $H\times K$ then 
$$L_H = \{h\in H | (h,k)\in L \mbox{ for some }k\in K\}$$ 
is a subgroup of $H$. 
Similarly 
$$L_K = \{k\in H | (h,k)\in L \mbox{ for some }h\in H\}$$ is a subgroup
of $K$.

The group $L^*=\{(h,k)|h\in L_H,k\in L_K\}$ is a subgroup of $H\times K$.
It evidently contains $L$ but can be bigger!
\item $G=H\times K$ contains two subgroups $H_1,K_1$ isomorphic to $H$
and $K$ respectively. namely $H_1=\{ (h,1)|h\in H\}$ and
$K_1=\{(1,k)|k\in K\}$. Note that we are using the symbol ``$1$'' to
denote the identity in both the groups. This is an {\bf abuse of
notation,}
but helps us keep the notation short.
\item
We shall name the natural projection map $\pi_H:G\rightarrow H$ defined
by $\pi_H(h,k) = h$. Similarly $\pi_K(h,k)=k$.

Note that then $L_H=\pi_H(L)$ and $L_K=\pi_K(L)$.
Both projections are group homomorphisms with $Ker(\pi_H)=K_1$ and
$Ker(\pi_K)=H_1$. Thus $G/H_1\cong K$ and $G/K_1\cong H$.

\item {\bf Internal direct product.}\index{group!product!internal
direct}
Note that $H_1,K_1$ are subgroups of $G$ satisfying these properties:

\begin{itemize}
\item $H_1\bigcap K_1 = \{1\}$. 
{\bf Convention:} Note that this $1$ is really the
identity $(1,1)\in G$. We may often just write $1$ for $\{1\}$ in order
to denote the appropriate group containing just the appropriate
identity, or the so-called identity subgroup.

\item $G=H_1K_1$.
\item $[H_1,K_1]=1$. This cryptic notation is saying that elements of
$H_1,K_1$ commute with each other and hence their commutators reduce to
the identity subgroup.

\end{itemize}
We shall say that the group $G$ is an internal direct product of
$H_1,K_1$ if the above conditions hold. 

It can be deduced from the given conditions that every element 
$g \in G$ is {\bf uniquely} a product of some $h_1\in H_1$ and $k_1\in K_1$ so that 
$g=h_1k_1$. As a result, it is also easy to prove that an internal
direct product is in turn isomorphic to an external direct product. We
simply take the map $H_1\times K_1 \rightarrow G$ defined by
$(h_1,k_1)\rightarrow h_1k_1$.

{\bf Example.} Let $G=\DZ_{15}$ and $H_1=<[3]_{15}>$ and
$K_1=<[5]_{15}>$.
Check that $G$ is an internal direct product of $H_1,K_1$.
\footnote{Remember that we need to use the additive notation here.}

Note, in turn, that $H_1\cong \DZ_{5}$ and $K_1\cong \DZ_{3}$ and this
is simply the reaffirmation that $\DZ_{15}\cong \DZ_{5}\times \DZ_{3}$.

\item {\bf Semidirect product.}\index{group!product!semidirect}
Now consider the example of the dihedral group $D_{2n}$ with $n\geq 3$.

Let $H=<r>$ where $r$ is the usual element of order $n$ and $K=<s>$
where $s$ is the element of order $2$ satisfying $srs=r^{-1}$.
Then $H\bigcap K = 1$ and $D_{2n}=HK$ but $[H,K]\neq 1$.
The group $H$, however is a normal subgroup and $D_{2n}/H$ is isomorphic
to $\DZ_2 \cong K$.

We formalize a product concept to model this behavior.

Let $H,K$ be groups and $\phi:K\rightarrow Aut(H)$ a group
homomorphism. 
\deff{The semidirect product of $\mathbf{H}$ with $\mathbf{K}$
 induced
by $\mathbf{\phi}$ } is denoted by $H\rtimes_\phi K$ and is defined thus:
\begin{itemize}
\item $H\rtimes_\phi K = H \times K$, i.e. as a set it coincides with
the direct product.
\item The product (i.e. the group operation)  is defined by 
$$(h_1,k_1)(h_2,k_2)=(h_1\phi(k_1)(h_2),k_1k_2).$$

This is sometimes written by using the suggestive notation 
$h_2^{\phi(k_1)}$ for $\phi(k_1)(h_2)$. Thus we write:

$$(h_1,k_1)(h_2,k_2)=(h_1(h_2)^{\phi(k_1)},k_1k_2).$$

The idea is that $\phi(k_1)$ is an
automorphism of $H$ and hence we let it act on $h_2$.
\footnote{The book shortens the notation to $k_1\cdot h_2$. Some people
write it as $h_2^{k_1}$ suppressing the notation $\phi$. No matter how
you shorten it, it is hiding many calculations and hence needs to be
well defined and understood!}

\item Just as in the case of the direct product, we have subgroups
$H_1,K_1$ isomorphic to $H,K$ respectively. This time, only $H_1$ is 
expected to be a normal subgroup and we have only one projection map  
$\pi_H:H\rtimes_\phi K \rightarrow K$.
\footnote{The other projection map does exist, but won't be a group
homomorphism! You should check in $D_{2n}$.}

\item {\bf Convention.} We can drop the reference to $\phi$ in the
notation if it is clear from the context.

\end{itemize}

{\bf Example.} Thus in the $D_{2n}$ example,  we can take $\phi(s)$ to
be the automorphism of $H=<r>$ which takes $r$ to $r^{-1}$. For
convenience, denote this automorphism of $H$ by $\tau$.

Then the image $\phi(<s>)$ is simply $<\tau>$.
Thus we have $$D_{2n}\cong \DZ_n\rtimes_{\phi} \DZ_2$$
where, we have identified $<r>$ with $\DZ_n$, $<s>$ with $\DZ_2$
and $Aut(H)$ with $\DZ_n^\times$.


\end{enumerate}

{\bf Remark.}
\begin{enumerate}
\item We may think of the direct product as a special case of a
semidirect product by taking the trivial homomorphism $\phi$, namely
taking $\phi(k)$ to be the identity automorphism for all $k\in K$.

\item If $H\triangleleft G$ is a normal subgroup and $K<G$ is any
subgroup such that $H\bigcap K=1$ then we know that $HK$ is a subgroup
of $G$. We can identify it as a semidirect product by defining $\phi(k)$ 
to be the conjugation automorphism (inner automorphism) $\phi(k)(h)=khk^{-1}$.

It can be easily seen that then $H\rtimes_\phi K \cong HK$ with the
identification $(h,k)\rightarrow hk$. We only show the homomorphism
property.

We have:
$$(h_1,k_1)(h_2,k_2) = (h_1(k_1h_2k_1^{-1}),k_1k_2)$$ and the
homomorphism condition requires that
$$(h_1k_1)(h_2k_2) = (h_1(k_1h_2k_1^{-1}))(k_1k_2)$$
which is evident! The rest of the claim follows.
 
We can simply call this an {\bf internal semidirect product}
\index{group!product!internal semidirect} and write $HK$ as $H\rtimes
K$. Note that the map $\phi$ is not mentioned, if it corresponds to 
matching elements with the inner automorphisms induced by them. 


\end{enumerate}

\subsection{Some examples and exercises.}
Here are some examples with details left for the reader and some
exercisess to be completed.
\begin{enumerate}
\item {\bf Another view of an abelian group.} Let $G$ be a group and use
the usual notation $G'$ to denote its commutator subgroup $[G,G]$.

Prove that $G/G'$ is an abelian group. ({\it Remember that the notation
implies that $G'$ is a normal subgroup and this also needs a proof!})

Prove that if $G/H$ is abelian for some normal subgroup $H$, then
$G' \triangleleft H$. This is sometimes described as ``$G'$ is  the
smallest normal subgroup with abelian quotient.''

\item Prove that if $A,B$ are subgroups of $G$ such that $G/A$ and $G/B$
are abelian, then $G/(A\bigcap B)$ is also abelian.

\item Let $K$ be a cyclic group of order $n>1$.

Deduce from old exercises that $Aut(K)$ is isomorphic to $\DZ_n^\times$
- the multiplicative  group $\{[a]_n|\mbox{GCD}(a,n)=1\}$ of order $\phi(n)$.

In particular, note that $Aut(K)$ is abelian.

\item Let $K$ be a cyclic normal subgroup of $G$.
Prove that $G' \subset C_G(K)$.
{\bf Hint:} You need to prove that conjugation by any element of $G'$
is the identity automorphism of $K$. Use knowledge of the $Aut(K)$
above.


\item Prove that a group of order $p^2$ where $p$ is a prime is either
cyclic or isomorphic to $\DZ_p\times \DZ_p$. Thus, in either case, it is
abelian.

\item Let $G$ be a group of order $p^3$ where $p$ is a prime.
Recall the proof that $G$ has a non trivial center and deduce that there
is an element $z\in G$ of order $p$ such that $[z,G]=\{e\}.$

Conclude that the commutator subgroup $G'\subset <z>$.

\item Assume that the elements $x,y\in G$ commute with $[x,y]$.
Prove the formula:
$$(xy)^n = x^ny^n[y,x]^{\frac{n(n-1)}{2}}.$$
{\bf Hint:} Use induction on $n$.


\item Let $G$ be a group of order $p^3$ where $p$ is an odd prime.
Prove that the map $t\rightarrow t^{p}$ is a homomorphism of $G$ into
$Z(G)$.

{\it Why does $p$ have to be odd? What changes when $p=2$?}

\item Combine the above results to describe all groups of order $p^3$
where $p$ is an odd prime.

{\bf Hint:} Start with $z$ in the center as above and separately 
consider the cases
when $G/<z>$ is either cyclic or isomorphic to $\DZ_p \times \DZ_p$.


\end{enumerate}

\section{Further Theorems in Groups.}
We now turn our attention to some topics of importance in applications 
of Group Theory. We may not get to the actual applications for some time, 
though!

\subsection{Fundamentals of $p$-groups.}
Having studied the existence $p$-Sylow groups, we now turn our attention 
to the structure of the $p$-Sylow subgroups themselves.

First some definitions.

\deff{Maximal subgroup.} 

A subgroup $H<G$ is said to be maximal if
$H\neq G$ and there is no subgroup $K$ strictly between $H$ and $G$.

\deff{Characteristic subgroup.}\index{group!characteristic}
 
A subgroup $H<G$ is said to be
characteristic if for every automorphism $\sigma \in Aut(g)$ we have
$\sigma(H)=H$. We can alternatively describe this as 
`` every automorphism stabilizes $H$ ''.

We write $H~char~G$, if $H$ is a characteristic subgroup of $G$.

Note that a normal subgroup only requires that it is stabilized by all
inner automorphisms, so a characteristic subgroup is necessarily normal.

Here are some {\bf facts about characteristic subgroups} worthy of remembering:
\begin{itemize}
\item $\{e\}$ and $G$ are trivial characteristic subgroups.

The commutator subgroup $G'$ is also characteristic. The center $Z(G)$
is also characteristic.

\item If $G$ has a unique $p$-Sylow subgroup, then it is clearly 
characteristic. 

\item If $K~char~H$ and $H \triangleleft G$, then $K \triangleleft G$.
For proof, we simply note that the inner automorphism given by
conjugation by an element of $G$ induces an automorphism of $H$ since
$H$ is normal in $G$ and hence an automorphism of $K$ since $K$ is
characteristic in $H$.

\end{itemize}

Here are some {\bf important properties of $\mathbf{p}$-groups.}
Let $p$ be a prime.
Let $P$ be a $p$-group. The Sylow theorems don't say anything further
about it, since it is its own $p$-Sylow subgroup. So we need further
analysis to understand its properties.

\begin{enumerate}
\item $P$ has non trivial center i.e. $|Z(P)|> 1$.

This is already
established by considering the conjugation action of $P$ on itself and
invoking the class equation:
$$|P| = |Z(P)| + \mbox{\bf sum of lengths of non singleton orbits }.$$

Since the LHS and the second term of RHS are divisible by $p$, so is
$|Z(P)|$ and since $Z(P)$ contains at least the identity element,
$|Z(P)|$ is at least $p$, giving the result!

\item Every normal subgroup $H$ of $P$ also contains non trivial central
elements.

This is done by repeating the above argument, but this time letting $P$
act on $H$ by conjugation. (This is meaningful, since $H$ is normal!)
Thus we get:

$$|H| = |Z(P)\bigcap H| + \mbox{\bf sum of lengths of non singleton orbits }$$
and thus, as before $|Z(P)\bigcap H|>1$.

\item Every normal subgroup $H$ of $P$ of order $p^s$ contains subgroups
of order $p^i$, for $i=0,1,\cdots,s$ which are {\bf normal in
$\mathbf{P}$.}

To see this, note that we already know the result for $i=0,1$ since $H$
contains the identity as well as a nontrivial central element (which we
may assume to be of order $p$ by raising to a power if necessary).

By induction, let $0\leq i <s $ and assume that $H_i$ is a normal
subgroup of $P$ such that
it is a subgroup of $H$ and we have $|H_i|=p^i$.
 We will show that there is $H_{i+1}<H$ such that
$H_{i+1}\triangleleft P$ and $|H_{i+1}|=p^{i+1}$.

For proof, simply consider $K=H/H_i < P/H_i$ and note that
$|H/H_i|=p^{s-i}>1$. Hence, we can choose an element $z\in H$ such that
its image $\overline{z}$ in $H/H_i$ is a non trivial central element of
$P/H_i$. Further arrange that $|\overline{z}|=p$ by raising it to a
suitable power if necessary.

With a little work, we see that $H_{i+1} =< H_i,z>$, satisfies all the
needed conditions.

\item In particular, $P$ contains {\bf normal subgroups} of all orders
dividing $|P|$.

We simply apply the above to $H=P$. The main point is that a sequence of
elements $z_1,z_2,\cdots,z_j$ generates a normal subgroup if each $z_i$
is central modulo the subgroup generated by $z_1,\cdots,z_{i-1}$. It is
this fact that leads to the idea of studying upper central series.

\item For any {\it proper subgroup}  $H$ of $P$, we have that
$N_P(H)$ is strictly bigger than $H$.

We make an induction on $|P|$. If $|P|=1$ or $p$ or even $p^2$, then the
group is abelian and there is nothing to prove.

We have two cases.
\begin{itemize}
\item {\bf Case 1.} Suppose $Z(P)$ contains an element $x \not\in H$.
Then $x\in N_P(H)\setminus H$ and the proof is finished!

\item {\bf Case 2.} Suppose $Z(P)\subset H$. Then we go modulo $Z(P)$
and note that $|P/Z(P)|<|P|$, so induction hypothesis can be applied to
the group $H/Z(P)$.

Write $\overline{P}=P/Z(P)$ and $\overline{H}=H/Z(P)$.

Thus, there is $\overline{x} \in \overline{P}$ such that
$\overline{x}\not\in \overline{H}$ but
$\overline{x}\in N_{\overline{P}}(\overline{H})$.

It is easy to check that $x$ is then in $N_P(H)\setminus H$.

\end{itemize}

\item Every maximal subgroup $Q$ of $P$ is normal. Moreover, such a maximal
$Q$ has index $p$.

This is immediate from the above, since the normalizer $N_P(Q)$ can only
be $P$ by maximality of $Q$, so $Q$ is normal! Moreover, the $p$-group
$P/Q$ has no proper subgroups and hence must be the smallest non trivial
$p$-group, namely the cyclic one of order $p$.



\end{enumerate}

\subsection{Fundamental Theorem of finite abelian groups.}
\index{Theorem!Fundamental theorem of finite abelian groups}

The fundamental theorem of finitely generated
abelian groups states that any finitely
generated abelian group $A$ is isomorphic to a direct product of cyclic
abelian groups. Note that a finite cyclic group is isomorphic to $DZ_n$
for some $n$, whereas an infinite cyclic group is isomorphic to $\DZ$.

There are further uniqueness statements associated with this theorem.
\begin{enumerate}
\item The number of infinite cyclic groups appearing in the direct
product only depends on the group and is called its {\bf free rank} or
{\bf Betti number}.
\item The product representation of the finite cyclic groups is not
unique, unless one makes some restrictions. One way is to require that
all cyclic factors be $p$-groups for various primes $p$ and then the
representation is unique except for order.

It is customary to use a more natural decomposition in terms of the
so-called invariant factors, but since it can be uniquely deduced from
the $p$-group factorization, we postpone its discussion until it comes
up naturally in a more general context (where $p$-group concept does not
generalize well).

\end{enumerate}

In this section, we only prove the version for finite groups, so the
free rank is $0$ and no copies of $\DZ$ appear!

We make a

\deff{Exponent of a group.} A positive integer $m$ is said to be the
exponent of a group $G$ if $g^m=e$ for each $g\in G$ and that $m$
is the least such integer.
\index{group!exponent of}

If such an integer does not exist, then we define it to be $\infty$.
It is clearly the LCM of the orders of elements of $G$.

Note that a finite group always has a finite exponent and it is a factor of
$|G|$, in view of Lagrange's Theorem.

Let us note that in the additive notation of abelian groups, we rewrite
the condition as $mg=0$.

{\bf Preamble for the Fundamental Theorem of finite abelian groups.}
Let $A$ be a finite abelian group of order $n$ and exponent $m |n$.
We shall assume that $p_1,\cdots,p_r$ are the different prime factors of
$n$

Thus, the theorem we aim  to prove is that:

$$A = \prod_{i=1}^r A_i \mbox{ where $A_i$ is a $p_i$-group.}$$
Here we are using the internal direct product.

Further, if we let $n_i=\ord_{p_i}(n)$ then $A_i$ is the unique normal
$p_i$-Sylow subgroup of $A$ of order $p_i^{n_i}$ and itself is congruent
to a product of $p_i$-groups, $A_{i_1},\cdots$.

\begin{enumerate}
\item
{\bf Reduction to the $\mathbf{p}$-group case.}
We prove a

{\bf Lemma.} Suppose that the  exponent $m$ of $A$ has a factorization
$m=ab$ where $\mbox{GCD}(a,b)=1$, then
$$A = H \times K$$
where $H$ and $K$ have exponents  $a,b$ respectively. Here, we are
actually claiming that $A$ is an internal direct product of $H,K$.


{\bf Proof.}
Define $H=\{x\in A | x^a=e\}$ and $K=\{x\in A | x^b=e\}$.
Using the abelian nature of $A$ we see that $H,K$ are subgroups.
\footnote{Simplest precise argument might be to note that $H$ is simply
the kernel of the $a$-power homomorphism $F_a$ from $A$ into $A$.
Similarly for $K$.}

Since $a,b$ are coprime, there exist integers $u,v$ such that $au+bv=1$.

If $x\in H\bigcap K$ then from $x^a=x^b=e$ we note that
$$x=x^1 = x^{au+bv} = (x^a)^u(x^b)^v =e.$$
Thus $H\bigcap K = \{e\}$.

Moreover, given any $x\in A$, note that $x^b\in H$ since clearly
$(x^b)^a = x^m =e$. Similarly, $x^a\in K$ since $(x^a)^b = x^m =e$.

Thus, by the same calculation as above $x=(x^b)^v (x^a)^u \in HK$. We
have thus proved that $A$ is the internal direct product of $H$ and $K$
as asserted. Clearly, the exponent of $H$ is some $a_1$ dividing $a$ and
the exponent of $K$ is some $b_1$ dividing $b$.

gFrom $A=HK$ it follows that $x^{a_1b_1}=e$ for all $x\in A$ and hence
$m=ab$ divides $a_1b_1$. From coprimeness of $a,b$ it follows that
$a_1=a$ and $b_1=b$.




{\bf The reduction.}
Applying the above lemma repeatedly, it is easy to see that the group
$A$ is isomorphic to the direct product of groups $A_i$ whose orders are
powers of a single prime $p_i$.
This reduces the proof of the theorem to the case where we only need to
consider $a$ to be a $p$-group for some prime $p$.

\item {\bf Case of a $\mathbf{p}$-group.}
Now we {\bf simplify our notation} by assuming that $p$ is a prime and
$A$ is a finite abelian $p$-group of order $n$ and exponent $m$.

{\bf Important change of representation.}
We will find it more convenient to think of our abelian group in
additive notation.

We now list the changes in our language caused by this conversion.
\begin{itemize}
\item The identity of the group $A$ is now $0$ and the exponent $m$ is
the smallest positive integer such that $mx=0$ for all $x\in A$.

\item We shall use induction on $n$, the case of $n=1,p$ being trivial.


\item {\bf The elementary abelian case.}\index{group!Elementary abelian}
An abelian $p$-group $A$ is said to be elementary abelian, if every non zero
element (non identity element) has order  $p$, i.e. $pA=\{0\}$.

Another way of saying this is to note that equivalently, the exponent is $p$.

Drawing on our knowledge from elementary algebra and linear algebra, we
note that such a group is naturally a {\it vector space} over the field
$\DZ_p$. We use the usual additive group structure and the scalar
multiplication $[r]_px$ when $r\in \DZ_p$ is simply defined as $rx$. The
condition of being elementary abelian makes this well defined.

Now, the usual basis of the vector space gives the generators necessary
to make the direct product (or direct sum in conformity with the
additive notation).

Thus, if $x_1,\cdots,x_t$ is a basis of $A$, then
$$A = \prod_{i=1}^t <x_i>.$$

This is easy to see from the vector space theory. Thus our theorem is
now considered proved for elementary abelian $p$-groups.
\end{itemize}

We shall prove that $A$ is a product of cyclic $p$-groups( or the so-called
direct sum in conformity with the additive notation.)


\item {\bf General case of the $\mathbf{p}$-group.}
Let $H=pA = \{px | x\in A\}$ and $K=\{x\in A | px=0\}$.

Note that the homomorphism $x\rightarrow px$ has kernel $K$ and image
$H$, so we get that $|A|=|K||H|$.

It is easy to see that $|H|<|A|$ since otherwise we will get an infinite
sequence of elements $x_1,\cdot,x_s,\cdots$ such that $x_i=px_{i+1}$, in
contradiction with the finiteness of $A$.

Thus, we may assume $H$ to be a direct product of cyclic $p$-groups
$<h_i>$, for $i=1,\cdots,r$, say. For each $h_i$ we can find $g_i\in A$
such that $pg_i=h_i$.

Then we can easily see that the group generated by $\{g_1,\cdots,g_r\}$
is the internal direct product of $\{<g_i>\}$ and we shall denote it by
$A_0$.

Suppose that $|h_i|=p^{n_i}$ for each $i$. Then it is easy to see from the
direct product condition that $|H|=\prod_{i=1}^r (p^{n_i})$ and
$|A_0|=\prod_{i=1}^r (p^{n_i+1})=|H|p^r$.

Moreover, clearly,
$$H\bigcap K= <{p^{n_1-1}}h_1>\times \cdots \times
<{p^{n_r-1}} h_r >.$$
Thus $|H\bigcap K| = p\cdots p = p^r$.

Now, if $K\subset H$ then
$$|K|=|H\bigcap K| =p^r$$
and we see that $|A_0| = |H|p^r = |H||K| = |A|$.
Then $A_0=A$ and the direct product structure of $A_0$ finishes the
proof.

In case $K\not\in H$ we need to enlarge $A_0$. Actually, in this case,
we invoke the induction in a different attack.

Consider the group $A/H$ which is clearly elementary abelian by
definition of $H$. Let $x\in K \setminus H$ and note that
$\overline{x}\in A/H$ is a non zero element. We can make a basis for the
$\DZ_p$-vector space $A/H$ as
$\overline{x}=\overline{y_1},\cdots,\overline{y_s}$.

We claim that $$A=<x>\times <y_2,\cdots,y_s>.$$
If we have an element in the intersection of $<x>$ and $
<y_2,\cdots,y_s>$, then it must be also in $H$, since it goes to zero
modulo $H$.
But $<x>\bigcap H = \{0\}$ and hence the intersection of the two groups
is the zero group.

Let $B$ be the group generated by $<x>$ and $ <y_2,\cdots,y_s>$.
It remains to show that every element $g\in A$ can be written as an
element of $B$. Clearly, we can write $g=u_1+pg_1$ where $u\in B$ and
$pg_1\in H$. Applying a similar process to $g_1$ and collecting terms,
we can write $g=u_1+pu_2+p^2g_2$. Continuing, we see write 
$$g=u_1+pu_2+\cdots+p^{i-1}u_{i}+p^{i} g_i.$$
For large enough $i$, we will have $p^i$ divisible by the exponent of
$A$ and hence $p^{i}g_{i}=0$. Thus $g=u_1+pu_2+\cdots+p^{i-1}u_i\in B$.

\item {\bf Uniqueness considerations.} We have now shown that for any
abelian $p$-group $A$, we have a sequence of non negative integers $d_i,
i=1,\cdots,t$  such that $A$ is a direct product of $d_1$ copies of
$\DZ_p$, $d_2$ copies of $\DZ_{p^2}$ and so on until $d_t$ copies of
$\DZ_{p^t}$, where $d_t\neq 0$.

We now explain how these numbers $d_i$ are uniquely determined by the
group.

\begin{itemize}
\item Assume that $A$ has exponent $p^m$.

Consider a decreasing sequence of groups $A_1=A,
A_2=pA_1,A_3=pA_2,\cdots$. Clearly $A_{m+1}=\{0\}$.



\item
Consider the successive quotient groups $B_i=A_i/A_{i+1}$ for
$i=1,\cdots m$. Note that $pB_i=\{0\}$ and hence each $B_i$ can be
thought of as a $\DZ_p$ vector space.

Let the dimensions of these vector
spaces be respectively $e_1,e_2,\cdots,e_m$.

We claim that the number of spaces isomorphic to copies of $\DZ_{p^m}$
is exactly $e_m$, i.e. $d_m=e_m$.

\item The number of spaces isomorphic to $\DZ_{p^{m-1}}$ is
$e_{m-1}-e_m$. Thus $d_{m-1}=e_{m-1}-e_m$.

Continuing, we get that $d_i=e_i-e_{i+1}$.

The idea of the proof is that a basis of $B_{i+1}$ is from classes of
elements from $A_{i+1}$ modulo $pA_{i+1}$, say
$\overline{x_1},\cdots,\overline{x_{e_{i+1}}}$.

From its definition, each $x_i$ is of the form $py_i$ where $y_i\in
A_i$. It can be argued that the elements
$\overline{y_1},\cdots,\overline{y_{e_{i+1}}}$ are independent in $B_i$,
so their number $e_{i+1}$ is less than or equal to the dimension of
$B_i$.

\end{itemize}
\end{enumerate}

\rightline{ To be continued $\dots$}
\vspace*{\fill}
\printindex

\end{document}