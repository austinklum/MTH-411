%This is a Latex file.
\documentclass[11pt]{article}
\usepackage{amsmath,amsfonts,amsthm}
\usepackage{dsfont}
\usepackage{fancyhdr}
\usepackage[margin=1in]{geometry}
\usepackage{lastpage} % Required to determine the last page for the footer
\usepackage{tikz}
\usepackage{url,hyperref}

\theoremstyle{plain}
\theoremstyle{definition}
\newtheorem{exercise}{Exercise}

\pagestyle{fancy} 
\lhead{\sf Austin Klum} \chead{\sf Reflection\#05} \rhead{\sf Fall 2018 MTH 411} 
\lfoot{} \cfoot{\thepage} \rfoot{}

%% See Page 3 of Fraleigh
\newcommand{\C}{\mathbb{C}}
\newcommand{\R}{\mathbb{R}}
\newcommand{\Q}{\mathbb{Q}}
\newcommand{\Z}{\mathbb{Z}}

\newcommand{\Rplus}{\mathbb{R^+}}
\newcommand{\Qplus}{\mathbb{Q^+}}
\newcommand{\Zplus}{\mathbb{Z^+}}

\newcommand{\Cstar}{\mathbb{C^*}}
\newcommand{\Rstar}{\mathbb{R^*}}
\newcommand{\Qstar}{\mathbb{Q^*}}
\newcommand{\Zstar}{\mathbb{Z^*}}



\begin{document}

Your weekly reflections are due every Sunday 11:59pm in the appropriate D2L Dropbox. They should consist of two sections (each of which can be further divided into subsections) as described in detail at the start of lecture 2.

\section{Reflection/\href{https://en.wikipedia.org/wiki/Logbook}{Logbook}/Diary}

\begin{itemize}
\item What did you find neat/cool/beautiful/incredible/\dots?
\item Generate questions/conjectures based on your reading of the text, my notes, and other sources (online, library, faculty, peers).
\item Try and answer some of them! 
\item Strive to discover connections with the math from your past.\\
Here are some sample sentence starters: ``This reminds me of \dots from class \dots \dots because \dots'' and ``I used to think \dots but now I think \dots because \dots''.

\end{itemize}

\section{Letter to Dr. Fraleigh}

\begin{itemize}
\item Comments based on your reading of Fraleigh. 
\item What did you find neat/cool/beautiful/incredible/\dots?
\item Comments on clarity or lack thereof.
\item Suggestions for improvement. This can be retroactive, based on future understanding and ``AHA'' moments!
\item Did you find any typos? Were there any words or symbols used that did not make sense? Did you figure out what those words meant (either by looking through the index, or pursuing an external resource).
\end{itemize}

\clearpage

\section{Reflection/Logbook/Diary}
The first theorem stating \textit{Every cylic group is abelian} was interesting to see as we had proved earlier in the section 4 exercises this very statement. So being able to compare and see how Fraleigh saw best to explain his proof was intriguing. The division algorithm was exciting to see again as the algorithm has been covered extensively in my number theory class this semester. It's really neat to see the division algorithm being used to show that every subgroup of a cyclic group is itself cyclic. I find this really fascinating and fairly intuitive. The only case that takes a bit of thought is the trivial case, which just cycles back onto itself. Another cool concept is the idea of finite cyclic groups and infinite cyclic groups. Some examples of finite groups are $ \Z_n $ and $ U_n $. While some examples of infinite cyclic groups are $ \Z $ and $ n\Z $ .Finite groups will cycle back on itself with only a set number of elements. This makes them easier to see when cycling through the possible values. While infinite cyclic groups have the properties of being cyclic, but are less obviously so. While, with finite cyclic groups you can see them wrapping around back to the start. In a infinite cyclic group you cannot see as easily that it will indeed "wrap" back onto itself.

\section{Letter to Fraleigh}
Dear Dr. Fraleigh,\\
\\
I found by proving that \textit{Every cyclic group is abelian} earlier in the text helpful in understanding your version of the proof. Being able to struggle through the proof and start making connections made the next chapter easier to understand. I found some of the definitions early in the chapter harder to understand and apply. The language surrounding \textbf{cyclic subgroup} $ \langle a \rangle $ \textbf{of} $ G $ \textbf{generated by} $ a $ is overly wordy and burdensome. I find the result that every subgroup of a cyclic group is cyclic itself a highly fascinating theorem. It's amazing how you can split up the cyclic group and still have cyclic groups. I found the corollary that shows how you can quickly find the other generators of a group $ G $ by finding the elements in $ G $ such that the order and the element are relatively prime to each other to be very useful and interesting. One criticisms I would have with the previous idea is not going deeper into the explanation of why such a result occurs.
\\
Sincerely,\\
Austin Klum


\end{document}
