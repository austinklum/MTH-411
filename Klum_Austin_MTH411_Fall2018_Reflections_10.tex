%This is a Latex file.
\documentclass[11pt]{article}
\usepackage{amsmath,amsfonts,amsthm}
\usepackage{dsfont}
\usepackage{fancyhdr}
\usepackage[margin=1in]{geometry}
\usepackage{lastpage} % Required to determine the last page for the footer
\usepackage{tikz}
\usepackage{url,hyperref}

\theoremstyle{plain}
\theoremstyle{definition}
\newtheorem{exercise}{Exercise}

\pagestyle{fancy} 
\lhead{\sf Austin Klum} \chead{\sf Reflection\#10} \rhead{\sf Fall 2018 MTH 411} 
\lfoot{} \cfoot{\thepage} \rfoot{}

%% See Page 3 of Fraleigh
\newcommand{\C}{\mathbb{C}}
\newcommand{\R}{\mathbb{R}}
\newcommand{\Q}{\mathbb{Q}}
\newcommand{\Z}{\mathbb{Z}}

\newcommand{\Rplus}{\mathbb{R^+}}
\newcommand{\Qplus}{\mathbb{Q^+}}
\newcommand{\Zplus}{\mathbb{Z^+}}

\newcommand{\Cstar}{\mathbb{C^*}}
\newcommand{\Rstar}{\mathbb{R^*}}
\newcommand{\Qstar}{\mathbb{Q^*}}
\newcommand{\Zstar}{\mathbb{Z^*}}



\begin{document}

Your weekly reflections are due every Sunday 11:59pm in the appropriate D2L Dropbox. They should consist of two sections (each of which can be further divided into subsections) as described in detail at the start of lecture 2.

\section{Reflection/\href{https://en.wikipedia.org/wiki/Logbook}{Logbook}/Diary}

\begin{itemize}
\item What did you find neat/cool/beautiful/incredible/\dots?
\item Generate questions/conjectures based on your reading of the text, my notes, and other sources (online, library, faculty, peers).
\item Try and answer some of them! 
\item Strive to discover connections with the math from your past.\\
Here are some sample sentence starters: ``This reminds me of \dots from class \dots \dots because \dots'' and ``I used to think \dots but now I think \dots because \dots''.

\end{itemize}

\section{Letter to Dr. Fraleigh}

\begin{itemize}
\item Comments based on your reading of Fraleigh. 
\item What did you find neat/cool/beautiful/incredible/\dots?
\item Comments on clarity or lack thereof.
\item Suggestions for improvement. This can be retroactive, based on future understanding and ``AHA'' moments!
\item Did you find any typos? Were there any words or symbols used that did not make sense? Did you figure out what those words meant (either by looking through the index, or pursuing an external resource).
\end{itemize}

\clearpage

\section{Reflection/Logbook/Diary}
I think it's really interesting that Lagrange was able to identify how the order of a subgroup of a finite group is always a divisor of the order of finite group. Cosets help fill in the missing link between groups and subgroups. There appears to be a connection between cosets and the division algorithm. The partitions are split up from 0 to n-1. I found the shading of the cayley table to be very intriguing. It's fascinating to see the smaller subsets of the table relates to other groups we have discussed. I am curious as to how the pattern forms for these subgroups. It would be exciting to find a an algorithmic approach to finding the subgroups in the table. I found the line stating \textit{Never underestimate results that count something!} to be humorous at first and began making sense once I read and thought more about it. The notation for cosets is fairly intuitive and easy enough to understand. I'm a little confused on what will happen when the group has infinite order. The Cartesian product of n-tuples reminds me of database theory with joins between two or more database tables. I find corollary 11.6 stated a cool fact, showing if the order of the groups are relatively prime, the group must be cyclic. This is a fairly quick method to use to test if the Cartesian products between groups are cyclic. The Fundamental Theorem of Finitely Generated Abelian Groups is essentially same as the Fundamental Theorem of Arithmetic both stating how the larger object can be split into smaller base objects. Its very cool how this idea of splitting up a large thing into smaller things is common throughout mathematics.

\section{Letter to Fraleigh}
Dear Dr. Fraleigh,\\
Overall I felt like chapter 10 and 11 were easy to understand and had excellent examples and clear explanations. I really appreciate the opening paragraph reviewing the the groups we described and have discussed in the previous sections. I wish you weren't sloppy in your notation. I have a hard time picking the correct usage of the notation in context. I feel like having a fixed notation that is standardized and use the same way every time would help make concepts easier to identify. I found the examples in this section to be very helpful. Often times I feel like the are several steps missing when completing the work and I get left behind on these small leaps of logic. In section 11, I thought the steps were broken down into manageable bite size chucks. I found the historical note to be reassuring. Even though Guass and Kummer were well know mathematicians who accomplished amazing things, it took a long time before someone came along and described these ideas in a more abstract way. 
\\
Sincerely,\\
Austin Klum


\end{document}
