%This is a Latex file.
\documentclass[11pt]{article}
\usepackage{amsmath,amsfonts,amsthm}
\usepackage{dsfont}
\usepackage{fancyhdr}
\usepackage[margin=1in]{geometry}
\usepackage{lastpage} % Required to determine the last page for the footer
\usepackage{tikz}
\usepackage{url,hyperref}

\theoremstyle{plain}
\theoremstyle{definition}
\newtheorem{exercise}{Exercise}

\pagestyle{fancy} 
\lhead{\sf Name} \chead{\sf Reflection\#X} \rhead{\sf Fall 2018 MTH 411} 
\lfoot{} \cfoot{\thepage} \rfoot{}

%% See Page 3 of Fraleigh
\newcommand{\C}{\mathbb{C}}
\newcommand{\R}{\mathbb{R}}
\newcommand{\Q}{\mathbb{Q}}
\newcommand{\Z}{\mathbb{Z}}

\newcommand{\Rplus}{\mathbb{R^+}}
\newcommand{\Qplus}{\mathbb{Q^+}}
\newcommand{\Zplus}{\mathbb{Z^+}}

\newcommand{\Cstar}{\mathbb{C^*}}
\newcommand{\Rstar}{\mathbb{R^*}}
\newcommand{\Qstar}{\mathbb{Q^*}}
\newcommand{\Zstar}{\mathbb{Z^*}}



\begin{document}

Your weekly reflections are due every Sunday 11:59pm in the appropriate D2L Dropbox. They should consist of two sections (each of which can be further divided into subsections) as described in detail at the start of lecture 2.

\section{Reflection/\href{https://en.wikipedia.org/wiki/Logbook}{Logbook}/Diary}

\begin{itemize}
\item What did you find neat/cool/beautiful/incredible/\dots?
\item Generate questions/conjectures based on your reading of the text, my notes, and other sources (online, library, faculty, peers).
\item Try and answer some of them! 
\item Strive to discover connections with the math from your past.\\
Here are some sample sentence starters: ``This reminds me of \dots from class \dots \dots because \dots'' and ``I used to think \dots but now I think \dots because \dots''.

\end{itemize}

\section{Letter to Dr. Fraleigh}

\begin{itemize}
\item Comments based on your reading of Fraleigh. 
\item What did you find neat/cool/beautiful/incredible/\dots?
\item Comments on clarity or lack thereof.
\item Suggestions for improvement. This can be retroactive, based on future understanding and ``AHA'' moments!
\item Did you find any typos? Were there any words or symbols used that did not make sense? Did you figure out what those words meant (either by looking through the index, or pursuing an external resource).
\end{itemize}

\clearpage

\section{Reflection/Logbook/Diary}
I found the addition and multiplication notation very convenient, but could lead to potential issues. In most cases, the operations being performed are clear, but bring up some ambiguity. Another interesting idea is the subgroup diagram. The diagrams make understanding how the subgroup connect with one another easier and more visually oriented. I'm a little confused on how some of the levels of graph relate to the other subgroups. Cyclic Subgroups are interesting as they are define as $ H = \{a^n|n\in\Z\} $ for all $ a\in $ a group $ G $. I'm wondering why cyclic groups are defined in such a way. Is there an intuition behind the definition? Why would such a way of describing cyclic groups be defined be as $ a^n $? It's also a cool fact that all cyclic groups are all abelian. The discussion on the division algorithm relates really well to my number theory class. We have been discussing how the division algorithm works and the concept of the $ \gcd $. 

\section{Letter to Fraleigh}
Dear Dr. Fraleigh,\\
\\
I enjoyed the additional reading on the history behind the mathematics being covered. I found it really fascinating the discovery of the current mathematics we take for granted as being truth. I would find asking the right questions to get an answer that's never been thought of before extremely difficult to do. I have a hard time guessing and forming my own postulates without reference to the book to keep me on track. Some potential confusion on \textbf{order} of a group would be confusing order of a group with the order of elements of a group. While the former is similar in concept to cardinality of set, the latter is more akin to sorting the elements in ascending or descending order. I would've liked the subgroup diagrams to be explained a little more throughly and how they represent the subgroups. I'm unsure if I like how the exercises will sometimes elaborate more or provide foreshadowing to the next section. I feel like overall it is a good way of introducing more material in an applicable way.
\\
Sincerely,\\
Austin Klum


\end{document}
