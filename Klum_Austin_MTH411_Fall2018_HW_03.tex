%This is a Latex file.
\documentclass[12pt]{article}
\usepackage{latexsym,fancyhdr,amsmath,amsfonts,dsfont,amsthm,amssymb,mathrsfs,mathtools}
\usepackage[margin=0.94in]{geometry}
\usepackage{lastpage} % Required to determine the last page for the footer
\usepackage{tikz}
\usepackage{url, hyperref}

\parindent 0pt

\pagestyle{fancy} \lhead{\sf MTH 411} \chead{\sf Homework \#03}
\rhead{\sf Due: Saturday 10/2/2018} \lfoot{} \cfoot{} \rfoot{}

\newcommand{\N}{\mathds{N}}
\newcommand{\Z}{\mathds{Z}}
\renewcommand{\vec}[1]{\overrightarrow{#1}}
\newcommand{\C}{\mathbb{C}}
\newcommand{\R}{\mathbb{R}}
\newcommand{\G}{\mathbb{G}}
\newcommand{\Q}{\mathbb{Q}}

\begin{document}

{\bf Reading}: \fbox{Sections 4 and 5.}

\

{\bf Section 4 Problems}: \underline{You should be able to work out all problems}, except 20-21 and 38. However, as mentioned in lectures, you should read/study all the problems (even ones you skip), so you (i) know the statements/results contained therein, and (ii) learn how the author generates questions!
%\begin{itemize}
%\item 
%\end{itemize}

\

{\bf Section 5 Problems}: \underline{You should be able to work out all problems}.

%\begin{itemize}
%\item 
%\end{itemize}

\hrulefill

\

The following problems are {\bf due on 11:59pm Tuesday 10/2/2018}.  Submit both LaTeX and pdf files to the appropriate D2L Dropbox. 

Please name the files using the following format:
\begin{center}
\fbox{LastName$\_$FirstName$\_$MTH411$\_$Fall2018$\_$HW$\_$3}
\end{center}

You may discuss the problems with your classmates, but your write-up must be your own.  Any problems marked with an asterisk (*) denote problems you can not discuss with anyone except for me.\\

Please include the statements of the problems in your HW submissions. For any Extra problems you can copy the statements from the LaTeX file that generated this pdf. However, you will have to transcribe the remaining problems from Fraleigh.

\

\textbf{Section 4}: 2, 4, 6, 10, 12, 18, 26, 32-35, 41\footnote{Hint for 34: Use the \href{https://en.wikipedia.org/wiki/Pigeonhole_principle}{The pigeon-hole principle}! I.e. if $n$ items/pigeons are put into $m$ containers/holes, with $n > m$, then at least one container must contain more than one item.}\\
\textbf{Section 5}: 8, 13, 22-25, 36, 37-38, 41, 44, 45, 52, 54, 55\\

%\textbf{Extras}:
%\begin{enumerate}
%\item
%\item
%\item
%\end{enumerate}

	\begin{enumerate}
		
		\item[Excersise 4.2-4.6]Determine whether the binary operation $*$ gives a group structure on the given set. If no group results, give the first axiom in the order $\mathscr{G}_1,\mathscr{G}_2, \mathscr{G}_3$ from Definition 4.1 that does not hold.
			\item[4.2] Let $*$ be defined on $2\Z = \{2n|n\in \Z\}$ by letting $a * b = a + b$.
			
			Let $a,b,c \in 2\Z$.
			Consider $a*(b* c)$. \[a* (b * c) = a + (b + c) = (a + b) + c = (a* b)* c\] Therefore $*$ is associative.
			
			Consider $0 = 2\cdot 0 \in 2 \Z$. Let $a \in 2\Z$. Then \[ 0 + a = a + 0 = a\]. Thus 0 is an identity element.
			
			Let $a \in 2\Z$. If $a = 2n$ for some $n\in \Z$, then $-n \in \Z$, so $-2n = -a \in 2\Z$. Observe \[a + (-a) = (-a) + a = 0\]
			Thus, for all $x \in 2\Z$ there exists an inverse element.
			Therefore, because $*$ is associative, there exists an identity element, and because for all $x \in 2\Z$ there exists an inverse element, $2\Z$ closed under addition is a group.
			
			\item[4.04] Let $*$ be defined on $\Q$ by letting $a* b = ab$
			\item[4.06] Let $*$ be defined on $\C$ by letting $a* b = |ab|$
			\item[4.10] Let $n$ be a positive integer and let $n\in\Z = \{nm | m\in Z\}$
			\begin{enumerate}
				\item 
				
				\item 
			\end{enumerate}
			For 4.12 and 4.18, determine whether the given set of matrices under the specified operation is a group.(additional hints in text)
			\item[4.12] All $n \times n$ diagonal matrices under matrix multiplication.
			\item[4.18] All $n \times n$ matrices with determinant either 1 or -1 under matrix multiplication.
			\item[4.26] Give a one sentence proof synopsis of the proof of the left cancellation law in Theorem 4.15
			\item[4.32] Show that every group $G$ with identity $e$ such that $x * x = e$ for all $x \in G$ is abelian.
			\item[4.33] Let $G$ be an abelian group and let $c^n = c* c * \dotsb * c $ for $n$ factors $c$, where $c\in G$ and $n \in \Z^+$. Give a mathematical induction proof that $(a * b)^n = (a^n) * (b^n)$ for al $a,b \in G$.
			\item[4.34] Let $G$ be a group with a finite number of elements. Show that for any $a \in G$, there exists an $n\in \Z^+$ such that $a^n = e$. (hints in text).
			\item[4.35] Show that if $(a  * b)^2 = a^2  * b^2$ for $a$ and $b$ in a group $G$, then $a * b = b  * a$.(text for hints). 
			\item[4.41] Let $G$ be a group and let $g$ be one fixed element of $G$. Show that the map $i_8$, such that $i_8(x) = gxg'$ for $x\in G$, is an isomorphism of $G$ with itself.
		
		In $5.8$ and $5.13$, determine whether the given set of inevitable $n \times n$ matrices with real number entries is a subgroup of $GL(n,\R)$.
			\item[5.8] The $n \times n$ matrices with no zeros on the diagonal
			\item[5.13] The set of all $n \times n$ matrices $A$ such that $(A^T)A=I_n$ (hints in text)
		In 5.22-5.25, find the order of the cyclic subgroup of $GL(2,\R)$ generated by the given $2 \times 2$ matrix. 
		
		\[ 22.
		\begin{bmatrix}
		0   & -1 \\
		-1   & 0 \\
		\end{bmatrix}
		23.
		\begin{bmatrix}
		1   & 1 \\
		0   & 1 \\
		\end{bmatrix}
		24.
		\begin{bmatrix}
		3   & 0 \\
		0   & 2 \\
		\end{bmatrix}
		25.
		\begin{bmatrix}
		0   & -2 \\
		-2   & 0 \\
		\end{bmatrix}
		\]
		\item[5.36]
		 \begin{enumerate}
			\item[a] Complete Table 5.25 to give the group $\Z_6$ of 6 elements.
			
			\item[b] Compute the subgroups <0>,<1>,<2>,<3>,<4> and <5> of the group $\Z_6$ given part $(a)$
			
			\item[c] Which elements are generators for the group $\Z_6$ of part $(a)$?
			
			\item[d] Give the subgroup diagram for the part $(b)$ subgroups of $\Z_6$.(text in text).
		\end{enumerate}
			
		For 5.37 and 5.38 correct the definition of the italicized term without reference to the text, if correction is needed.
		\item[5.37] A \textit{subgroup} of a group $G$ is a subset $H$ of $G$ that contains the identity element $e$ of $G$ and also contains the inverse of each of its elements.
		\item[5.38] A group $G$ is \textit{cyclic} if and only if here exists $a \in G$ such that $G = \{a^n|n\in\Z\}$.
		\item[5.41] Let $\phi: G \rightarrow G'$ be an isomorphism of a group $<G, *>$ with a group $<G', *'>$
		\item[5.44] Find the flaw in the following argument: "Condition 2 of Theorem 5.14 is redundant, since it can be derived from 1 and 3, for let $a \in H$. Then $a^{-1}\in H$ by 3, and by a, $aa^{-1} = e$ is an identity element of $H$, proving 2.
		\item[5.45] Show that a nonempty subset $H$ of a group $G$ is a subgroup of $G$ if and only if $ab^{-1}\in H$ for all $a,b \in H$.
		\item[5.52] Generalizing Exercise 51, let $S$ be any subset of a group $G$.
		
		\begin{enumerate}
			\item[a] Show that $H_s = \{x \in G|xs = sx \text{ for all} s \in S\}$ is a subgroup of $G$. 
			\item[b] In reference to part $(a)$, the subgroup $H_G$ is the \textbf{center of} $G$. Show that $H_G$ is an abelian group.
		\end{enumerate}
		
		\item[5.54] Show that if $H \leq G$ and $K \leq G$, then $H\cap K \leq G$.
		\item[5.55] Prove that every cyclic group is abelian.
		
	\end{enumerate}

\end{document}
