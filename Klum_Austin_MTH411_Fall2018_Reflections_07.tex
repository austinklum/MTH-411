%This is a Latex file.
\documentclass[11pt]{article}
\usepackage{amsmath,amsfonts,amsthm}
\usepackage{dsfont}
\usepackage{fancyhdr}
\usepackage[margin=1in]{geometry}
\usepackage{lastpage} % Required to determine the last page for the footer
\usepackage{tikz}
\usepackage{url,hyperref}

\theoremstyle{plain}
\theoremstyle{definition}
\newtheorem{exercise}{Exercise}

\pagestyle{fancy} 
\lhead{\sf Austin Klum} \chead{\sf Reflection\#05} \rhead{\sf Fall 2018 MTH 411} 
\lfoot{} \cfoot{\thepage} \rfoot{}

%% See Page 3 of Fraleigh
\newcommand{\C}{\mathbb{C}}
\newcommand{\R}{\mathbb{R}}
\newcommand{\Q}{\mathbb{Q}}
\newcommand{\Z}{\mathbb{Z}}

\newcommand{\Rplus}{\mathbb{R^+}}
\newcommand{\Qplus}{\mathbb{Q^+}}
\newcommand{\Zplus}{\mathbb{Z^+}}

\newcommand{\Cstar}{\mathbb{C^*}}
\newcommand{\Rstar}{\mathbb{R^*}}
\newcommand{\Qstar}{\mathbb{Q^*}}
\newcommand{\Zstar}{\mathbb{Z^*}}



\begin{document}

Your weekly reflections are due every Sunday 11:59pm in the appropriate D2L Dropbox. They should consist of two sections (each of which can be further divided into subsections) as described in detail at the start of lecture 2.

\section{Reflection/\href{https://en.wikipedia.org/wiki/Logbook}{Logbook}/Diary}

\begin{itemize}
\item What did you find neat/cool/beautiful/incredible/\dots?
\item Generate questions/conjectures based on your reading of the text, my notes, and other sources (online, library, faculty, peers).
\item Try and answer some of them! 
\item Strive to discover connections with the math from your past.\\
Here are some sample sentence starters: ``This reminds me of \dots from class \dots \dots because \dots'' and ``I used to think \dots but now I think \dots because \dots''.

\end{itemize}

\section{Letter to Dr. Fraleigh}

\begin{itemize}
\item Comments based on your reading of Fraleigh. 
\item What did you find neat/cool/beautiful/incredible/\dots?
\item Comments on clarity or lack thereof.
\item Suggestions for improvement. This can be retroactive, based on future understanding and ``AHA'' moments!
\item Did you find any typos? Were there any words or symbols used that did not make sense? Did you figure out what those words meant (either by looking through the index, or pursuing an external resource).
\end{itemize}

\clearpage

\section{Reflection/Logbook/Diary}
This weekend has been spent primarily doing math and preparing for the exam on Tuesday. I was able to study a good portion on Sunday with Dan, Brett and Alec. I've done a lot to prepare for this exam, but I still don't feel ready in the slightest. There's so much material and I cannot remember how to apply them all correctly. I'm sure the main ideas of the exam, I will do fine and have some inkling of what to do for each problem. It's the more minor stuff that I'm really concerned about. A bunch of little things wrong, adds up to a large sum. There's plenty of instances when we'll cover something once or twice and should be able to recall and execute the correct logic.\\
\\
\textit{Honestly, I would just skip or skim this section. It's mostly me griping about the semester...}\\
\\
 I'm really hoping this exam goes well. This semester has been really challenging for me and having a good score will help make life go easier. I just feel like I have so much work to get done, and that I do not understand very well on how to do said work. Normally, I can manage my time quite well and usually feel in complete control being able to juggle hitting up the gym, going to work, going to school, making sure to get sleep, having fun, and getting stuff done. This semester, I have to spend so much time in the homework category that I haven't been to the gym since week 2 and have been staying up late to 1 or 2am almost every night to complete work that's due the next day or two away. The worst part is struggling through the homework isn't productive for me. I have productive struggle in Computer Science all the time, where the project is difficult or challenging and through the work, I have a greater understanding of the subject and feel bettered prepared for the coursework. In my Math classes, I just stare at the page and will often have no idea what step to take next. So I'll end up looking online for hints, only to get stuck almost immediately after reading the next line in the proof. So, I'll read the next line for help, and so on, until most of the proof is just a poor attempt of me trying to translate the proof from online in my own words. Most of the time I can understand the steps others took to get the correct answer, but there are often times where I really have no idea what the proof is saying and end up just regurgitating a very similar proof, with little input from myself. Most of Fraleigh proofs I have a hard time understanding. 

\section{Letter to Fraleigh}
Dear Dr. Fraleigh,\\
I found defining permutations as simply functions or operations of permutation multiplication really interesting. This is a helpful way of thinking about permutations, rather than a unique ordering of elements in a set, and instead more in the view of everything else in this book, describing somethings in a more general way that can be applied in many various different situations. I would've liked more examples on how to do this operation, as the single example when brining up the subject did not make it clear as to exactly how to do the operation. Upon greater inspection over several times of re-reading, I was able to make some sense of how to complete such an operation. I really enjoyed the historical note. Often times mathematics is taught as if many of the ideas being covered are fairly obvious and intuitive. These historical notes show how many of the ideas we take for granted now, took a great deal of time to be developed. The topics taught in merely ten pages, took almost a thousand years to be where they are now.
\\
Sincerely,\\
Austin Klum


\end{document}
