%This is a Latex file.
\documentclass[12pt]{article}
\usepackage{latexsym,fancyhdr,amsmath,amsfonts,dsfont,amsthm,amssymb,mathrsfs,mathtools}
\usepackage[margin=0.94in]{geometry}
\usepackage{lastpage} % Required to determine the last page for the footer
\usepackage{tikz}
\usepackage{url, hyperref}

\parindent 0pt

\pagestyle{fancy} \lhead{\sf MTH 411} \chead{\sf Homework \#02}
\rhead{\sf Due: Tuesday 9/25/2018} \lfoot{} \cfoot{} \rfoot{}

\newcommand{\N}{\mathds{N}}
\newcommand{\Z}{\mathds{Z}}
\renewcommand{\vec}[1]{\overrightarrow{#1}}
\newcommand{\C}{\mathbb{C}}
\newcommand{\R}{\mathbb{R}}
\newcommand{\Q}{\mathbb{Q}}
\DeclarePairedDelimiter\abs{\lvert}{\rvert}

\begin{document}
	
	\begin{enumerate}
		\item[5]
		Suppose $ xH \cap yH \not = \emptyset $. Then $ \exists z \in xH \cap bH \Rightarrow \exists h_1,h_2\in H $ such that $ z = xh_1 $ and $ z=bh_2 $. Thus,
		 \[x=zh_1^{-1}=yh_2h_1^{-1} \text{ and } xH=yh_2h_1^{-1}H=y(h_2h_1^{-1}H)=yH\]
		 Therefore, $ xH = yH $
		 \item[6]
		 Suppose that $ x^{-1}y =h \in H$. Let $ xk \in xH $ with $ k \in H $. Then we have, $ xk = (yh^{-1})k = y(h^{-1}k)\in bH $. Thus, $ xH \subseteq yH $\\
		 The proof of $ yH\subseteq xH $ is similar.\\
		 Therefore, $ xH = yH $
	 	\end{enumerate}
\end{document}