%This is a Latex file.
\documentclass[11pt]{article}
\usepackage{amsmath,amsfonts,amsthm}
\usepackage{dsfont}
\usepackage{fancyhdr}
\usepackage[margin=1in]{geometry}
\usepackage{lastpage} % Required to determine the last page for the footer
\usepackage{tikz}
\usepackage{url,hyperref}

\theoremstyle{plain}
\theoremstyle{definition}
\newtheorem{exercise}{Exercise}

\pagestyle{fancy} 
\lhead{\sf Austin Klum} \chead{\sf Reflection\#05} \rhead{\sf Fall 2018 MTH 411} 
\lfoot{} \cfoot{\thepage} \rfoot{}

%% See Page 3 of Fraleigh
\newcommand{\C}{\mathbb{C}}
\newcommand{\R}{\mathbb{R}}
\newcommand{\Q}{\mathbb{Q}}
\newcommand{\Z}{\mathbb{Z}}

\newcommand{\Rplus}{\mathbb{R^+}}
\newcommand{\Qplus}{\mathbb{Q^+}}
\newcommand{\Zplus}{\mathbb{Z^+}}

\newcommand{\Cstar}{\mathbb{C^*}}
\newcommand{\Rstar}{\mathbb{R^*}}
\newcommand{\Qstar}{\mathbb{Q^*}}
\newcommand{\Zstar}{\mathbb{Z^*}}



\begin{document}

Your weekly reflections are due every Sunday 11:59pm in the appropriate D2L Dropbox. They should consist of two sections (each of which can be further divided into subsections) as described in detail at the start of lecture 2.

\section{Reflection/\href{https://en.wikipedia.org/wiki/Logbook}{Logbook}/Diary}

\begin{itemize}
\item What did you find neat/cool/beautiful/incredible/\dots?
\item Generate questions/conjectures based on your reading of the text, my notes, and other sources (online, library, faculty, peers).
\item Try and answer some of them! 
\item Strive to discover connections with the math from your past.\\
Here are some sample sentence starters: ``This reminds me of \dots from class \dots \dots because \dots'' and ``I used to think \dots but now I think \dots because \dots''.

\end{itemize}

\section{Letter to Dr. Fraleigh}

\begin{itemize}
\item Comments based on your reading of Fraleigh. 
\item What did you find neat/cool/beautiful/incredible/\dots?
\item Comments on clarity or lack thereof.
\item Suggestions for improvement. This can be retroactive, based on future understanding and ``AHA'' moments!
\item Did you find any typos? Were there any words or symbols used that did not make sense? Did you figure out what those words meant (either by looking through the index, or pursuing an external resource).
\end{itemize}

\clearpage

\section{Reflection/Logbook/Diary}
I found the section easy to understand in concept but difficult in application. I am familiar with ideas of permutations and some of their uses. I thought the concept of describing permutations as a bijective mapping was a great way of tying in previous ideas in this textbook. The notation was new to me, but seems to be straightforward enough. I found the use of apply the binary operation on the permutations a little difficult to understand. I am not completely sure how the mappings related to each other. I looked online to help further my understanding, yet I found little help to what I was actually looking for. I'm not sure what notation Fraleigh is using, but I wasn't able to find the same notation that's used in the text. I wasn't completely sure where the nth dihedral group falls into place upon initial reading. After further reading I was able to decipher what was meant by the group of symmetries of the regular n-gon. Cayley's Theorem was interesting to read. The application and uses are quite powerful. I'm not fully behind the explanation of why this theorem holds. Another idea that was hard to grasp was the left and right regular representation of G.


\section{Letter to Fraleigh}
Dear Dr. Fraleigh,\\
 The section was easy to understand at a high level, but once getting into the lower details and trying to use the theorem referred to in the textbook I had a hard time understanding how to use what was described. I enjoyed how the permutations were described in terms of a bijective mapping on itself. I would have liked to see more examples of permutation multiplication, as well as breaking the example up into smaller steps. I'm sure the steps are simple to understand once you know what's going on, but for me right now, it is too large of a leap of what's the process of computation. I really enjoyed the use of a historical note and how long these ideas were toyed and thought of before fully coming into full picture like we see the applications now. I found it interesting how several of the ideas that led to the development of this section were thought of so long ago that we no longer have the names of those who first thought of it. I liked the addition of the warning for when reading other texts about the ordering. I believe the language surrounding the ideas of Table 8.8 to be unclear and how they lead to the dihedral group and the symmetries of a group. I didn't like the language around saying how we "naively" to such and such a conclusion. Cayley's Theorem was a powerful theorem with numerous applications. I am not completely sure on the reasons why this theorem holds, but I can see how this would be an extremely useful theorem to use. I'm not sure what the left and right regular representation actually are doing. I wish there was a few more examples on how to use this idea.
\\
Sincerely,\\
Austin Klum


\end{document}
