%This is a Latex file.
\documentclass[11pt]{article}
\usepackage{amsmath,amsfonts,amsthm}
\usepackage{dsfont}
\usepackage{fancyhdr}
\usepackage[margin=1in]{geometry}
\usepackage{lastpage} % Required to determine the last page for the footer
\usepackage{tikz}
\usepackage{url,hyperref}

\theoremstyle{plain}
\theoremstyle{definition}
\newtheorem{exercise}{Exercise}

\pagestyle{fancy} 
\lhead{\sf Austin Klum} \chead{\sf Reflection\#09} \rhead{\sf Fall 2018 MTH 411} 
\lfoot{} \cfoot{\thepage} \rfoot{}

%% See Page 3 of Fraleigh
\newcommand{\C}{\mathbb{C}}
\newcommand{\R}{\mathbb{R}}
\newcommand{\Q}{\mathbb{Q}}
\newcommand{\Z}{\mathbb{Z}}

\newcommand{\Rplus}{\mathbb{R^+}}
\newcommand{\Qplus}{\mathbb{Q^+}}
\newcommand{\Zplus}{\mathbb{Z^+}}

\newcommand{\Cstar}{\mathbb{C^*}}
\newcommand{\Rstar}{\mathbb{R^*}}
\newcommand{\Qstar}{\mathbb{Q^*}}
\newcommand{\Zstar}{\mathbb{Z^*}}



\begin{document}

Your weekly reflections are due every Sunday 11:59pm in the appropriate D2L Dropbox. They should consist of two sections (each of which can be further divided into subsections) as described in detail at the start of lecture 2.

\section{Reflection/\href{https://en.wikipedia.org/wiki/Logbook}{Logbook}/Diary}

\begin{itemize}
\item What did you find neat/cool/beautiful/incredible/\dots?
\item Generate questions/conjectures based on your reading of the text, my notes, and other sources (online, library, faculty, peers).
\item Try and answer some of them! 
\item Strive to discover connections with the math from your past.\\
Here are some sample sentence starters: ``This reminds me of \dots from class \dots \dots because \dots'' and ``I used to think \dots but now I think \dots because \dots''.

\end{itemize}

\section{Letter to Dr. Fraleigh}

\begin{itemize}
\item Comments based on your reading of Fraleigh. 
\item What did you find neat/cool/beautiful/incredible/\dots?
\item Comments on clarity or lack thereof.
\item Suggestions for improvement. This can be retroactive, based on future understanding and ``AHA'' moments!
\item Did you find any typos? Were there any words or symbols used that did not make sense? Did you figure out what those words meant (either by looking through the index, or pursuing an external resource).
\end{itemize}

\clearpage

\section{Reflection/Logbook/Diary}
I felt like I understood what was meant by orbit conceptually. The orbit seems to be similar in nature to the generators of a cyclic subgroup. I am interested to see how this idea will develop in the further chapters and their uses. As we get further into the chapter we start discussing cycles and the usage of cyclic notation. I began have troubles following how the cyclic notation represents the earlier equation given. After some further study, I believe that the cyclic notation represents on the circle given only those values which get mapped someplace different, otherwise just map the numbers back onto themselves. The idea of every permutation of a finite set is a product of disjoint cycles reminds me of the fundamental theorem of arithmetic which describes how every composite number is composed of primes. The definition of a transposition is simplistic enough, but once the text began describing what the definition actually means and what can be accomplish, I had a harder time understanding what was happening. I'm not sure of the uses of the alternating group. The use for such an idea was described at the end of section 9, but I did not really understand the application. It seems to be stating how there is no simple quintic equation similar in nature to that of the quadratic equation.

\section{Letter to Fraleigh}
Dear Dr. Fraleigh,\\ 
 I felt section 9 was overall complied in a simple, easy to understand format with parallels to content covered earlier in the text. I wished you would have gone into more details and provided another example on how to use cyclic notation. I was not able to immediately identify the pattern based by the way the text described the notation. Only after several read throughs and processing was I able to determine the usage of cyclic notation. Transpositions were not defined clearly and the examples used were hard to follow. Also, who is David M. Bloom? Is he a reader that suggested the proof? Is he some famous mathematician? I felt like a little more context on why you mentioned him would have made more sense. I understood intuitively why no permutation can both expressed as an even and odd number of transpositions. Despite the intuitive nature of the theorem, I had a difficult time following the reasoning behind why such a statement was true. Describing even and odd on a permutation was simple enough, although I am not sure if I would be able to test if such a set was even or odd by splitting the set into transpositions. The alternating groups punchline seemed to be that there is no quintic equation. I was not really sure entirely what was going up until the final results.
\\
Sincerely,\\
Austin Klum


\end{document}
