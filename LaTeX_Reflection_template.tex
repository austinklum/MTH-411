%This is a Latex file.
\documentclass[11pt]{article}
\usepackage{amsmath,amsfonts,amsthm}
\usepackage{dsfont}
\usepackage{fancyhdr}
\usepackage[margin=1in]{geometry}
\usepackage{lastpage} % Required to determine the last page for the footer
\usepackage{tikz}
\usepackage{url,hyperref}

\theoremstyle{plain}
\theoremstyle{definition}
\newtheorem{exercise}{Exercise}

\pagestyle{fancy} 
\lhead{\sf Austin Klum} \chead{\sf Reflection\#01} \rhead{\sf Fall 2018 MTH 411} 
\lfoot{} \cfoot{\thepage} \rfoot{}

%% See Page 3 of Fraleigh
\newcommand{\C}{\mathbb{C}}
\newcommand{\R}{\mathbb{R}}
\newcommand{\Q}{\mathbb{Q}}
\newcommand{\Z}{\mathbb{Z}}

\newcommand{\Rplus}{\mathbb{R^+}}
\newcommand{\Qplus}{\mathbb{Q^+}}
\newcommand{\Zplus}{\mathbb{Z^+}}

\newcommand{\Cstar}{\mathbb{C^*}}
\newcommand{\Rstar}{\mathbb{R^*}}
\newcommand{\Qstar}{\mathbb{Q^*}}
\newcommand{\Zstar}{\mathbb{Z^*}}



\begin{document}

\section{Reflection/Logbook/Diary}
I thought the "Logic in a nutshell." table was very interesting. It's a good guide point to use for thinking about proofs. Most of this week has been connected directly to MTH-225. It's been good to revisit the material from before. I'm a little concerned as I'm not recalling everything as easily as I would've hoped. But as I'm taking both MTH 411 and 311. I am expecting to get lots of practice and expect to be ready for whatever is next.

\section{Letter to Dr. Fraleigh}
Dear Dr. Fraleigh,\\
\\
I found the few paragraphs of your \textit{A First Course in Abstract Algebra} to be fascinating. The beginning points out how writing a book of this nature can be difficult and cumbersome. How do you define everything without using any ideas that need to be defined in itself? As stated in the book, "It is impossible to define every concept." \\
\\
I had a difficult time understanding the \textbf{residue classes modulo} and the \textbf{congruence modulo n}. I had to reread the paragraphs several times and was still shaky on the entirety of it. I was able to reconnect the ideas to my computer science background and the use of the \textbf{mod} operator. In computer science the operator is used to find the remainder of a division of numbers. \\
\\
Sincerely,\\
Austin Klum
\end{document}
